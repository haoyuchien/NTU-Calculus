%
% Copyright 2018 Joel Feldman, Andrew Rechnitzer and Elyse Yeager.
% This work is licensed under a Creative Commons Attribution-NonCommercial-ShareAlike 4.0 International License.
% https://creativecommons.org/licenses/by-nc-sa/4.0/
%
\graphicspath{{figures/applications/}}

\chapter{Applications of Derivatives}
\section{Maximum and Minimum}
\begin{eg}
\begin{itemize}
\item $f(x) = \cos x$\\
$x \in R$ has infinity many local max and local min\\
abs max = $1$\\
abs min= $-1$
\item $f(x) = x^2$\\
local max: none\\
local min: $x= 0 \implies $ abs min\\
$f(x) = x^2 \geq 0$
\item $f(x) = x^3$\\
local max: none\\
local min: none
\end{itemize}
\end{eg}
\section{Extreme Value Theorem}
\begin{theorem}
Assume $f(x)$ is conti. on $[a, b]$, then\\
$f(x)$ has a abs max $f(C)$\\
$f(x)$ has a abs min $f(D)$\\
for some $C, D \in [a, b]$
\end{theorem}
\section{Fermat's Theorem}
\begin{theorem}
Assume
\begin{itemize}
\item[(1)] $f(x)$ has a local max or local min at $c$
\item[(2)] $f'(c)$ exists ($f$ is differentiable at $x = c$)
\end{itemize}
Then
$$f'(c) = 0$$
\end{theorem}
\begin{notn}
$\left\{ \begin{array}{ll}
           f'(c) \neq 0 \\
           f'(c) \text{exists}
        \end{array} \right.
$ $\implies f(c)$ is neither a local max nor a local min
\end{notn}
\begin{proof}
Fermat's Theorem
\begin{itemize}
\item[(1)] If $f$ has a local max at $c$\\
$\implies f(c+h) \leq f(c)$ if  $\mid h \mid$ is small enough
\item[(2)] If $f'(c)$ exists\\
$\implies \displaystyle f'(c) = \lim_{h \to 0} \frac{f(c+h)-f(c)}{h} = \lim_{h \to 0^+} \frac{f(c+h)-f(c)}{h} = \lim_{h \to 0^-} \frac{f(c+h)-f(c)}{h}$
\end{itemize}
$f(c+h) - f(c) \leq 0$ if $\mid h \mid$ is small enough
\begin{itemize}
\item if $h > 0$\\
$\begin{array}{rcl}
\implies & \displaystyle \frac{f(c+h) -f(c)}{h} \leq 0\\
\implies & \displaystyle \lim_{h \to 0^+} \frac{f(c+h) -f(c)}{h} \leq 0\\
\implies & \displaystyle \lim_{h \to 0} \frac{f(c+h) -f(c)}{h} \leq 0\\
\implies & f'(c) \leq 0
\end{array}$
\item if $h < 0$\\
$\begin{array}{rcl}
\implies & \displaystyle \frac{f(c+h) -f(c)}{h} \geq 0\\
\implies & \displaystyle \lim_{h \to 0^+} \frac{f(c+h) -f(c)}{h} \geq 0\\
\implies & \displaystyle \lim_{h \to 0} \frac{f(c+h) -f(c)}{h} \geq 0\\
\implies & f'(c) \geq 0
\end{array}$\\
\end{itemize}
$f'(c) \geq 0$ and $f'(c) \leq 0 \implies f'(c) =0$
\end{proof}

\begin{remark} \ 
\begin{itemize}
\item[(1)] Reverse of Fermat's Thm is \underline{false}\\
$f(x) = x^3 \implies f'(x) = 3x^2 \quad f'(0) =0$\\
$x=0$ is neither a local max nor a local min
\item[(2)]
$f(x) = \mid h \mid \implies f(x) > 0 \quad f(0) = 0$\\
$f'(0)$ doesn't exist\\
$0$ is a local min (abs min)
\end{itemize}
\end{remark}
\section{Critical Point (Critical Number)}
\begin{defn}
\begin{itemize}
\item If $f'(c)$ doesn't exist or $f''(c)$ = 0, then $c$ is called a critical point of $f$
\item If $f$ has a local max or a local min at $c$, then $c$ is a critical point of $f$
\end{itemize}
\end{defn}
\begin{eg}
Find critical point of $f(x) = x^{\frac{3}{5}}(4-x)$\\
$$\begin{array}{rcl}
f'(x) & = & \displaystyle \frac{3}{5} x^{-\frac{2}{5}}(4-x) + x^{\frac{3}{5}} \cdot -1\\
& = & \displaystyle \frac{1}{5}x^{-\frac{2}{5}}(12- 3x -5x)\\
& = & \displaystyle \frac{1}{5}x^{-\frac{2}{5}}(12-8x)\\
& = & \displaystyle \frac{1}{5} \cdot \frac{12-8x}{x^{\frac{2}{5}}}
\end{array}$$
let $f(x)'=0$
$$x = \displaystyle \frac{3}{2}$$
Find $x$ s.t. $f'(x)$ doesn't exist $\implies f'(0)$ doesn't exist
\end{eg}

\section{Rolle's Theorem}
\begin{theorem}
Assume
\begin{itemize}
\item[(H1)] $f(x)$ is conti. in $[a, b]$
\item[(H2)] $f(x)$ is differentiable in $(a, b)$
\item[(H3)] $f(a) =f(b)$
\end{itemize}
Then $\exists c \in (a, b) \text{ s.t } f'(c) =0$
\end{theorem}
\begin{remark}
等高兩點間必有波峰或波谷
\end{remark}


\begin{proof}
Rolle's Theorem\\
Without loss of generality, we may assume
$$f(a) = f(b) = 0$$
Otherwise let $g(x) = f(x) - f(a) = f(x) - f(b)$, then
$$\begin{array}{rcl}
g(a) = 0 & \text{ and } & g(b) = 0
\end{array}$$
3 cases:
\begin{itemize}
\item[(1)] $f(x) =0 \quad \forall x \in (a, b)\\
\implies f'(x) = 0 \quad  \forall x \in (a, b)$
\item[(2)] $\exists x \in (a, b)$ s.t. $f(x) > 0$\\
Extreme Value Thm, (H1) $\implies f$ has a local max at $c \in (a, b)$\\
Fermat's Thm, (H2) $\implies f'(c) = 0$
\item[(3)] $\exists x \in (a, b)$ s.t $f(x) < 0$\\
Extreme Value Thm, (H1) $\implies f$ has a local min at $c \in (a, b)$\\
Fermat's Thm, (H2) $\implies f'(c) = 0$
\end{itemize}
\end{proof}
\begin{eg}
Prove $x^3+x -1=0$ has one real root\\\\
let $f(x) = x^3+x-1$\\
$$\left\{ \begin{array}{ll}
           f(1) = 1 >0 \\
           f(-1) = -3 <0
        \end{array} \right.
$$
By I.V.T, $\exists c \in (-1, 1)$ s.t. $f(c) =0$\\
Assume $x_1$ and $x_2$ are two roots of $f(x)=0$\\
$$f(x_1)=f(x_2)=0$$
$$\left\{ \begin{array}{ll}
           \exists k \in (x_1, x_2) \  \text{s.t.} f(k)=0\\
           f'(x) = 3x^2 +1 \geq 1 > 0 \quad \forall x \in R
        \end{array} \right.
$$ 
$\implies$ contradiction
\end{eg}
\section{Mean Value Theorem}
\begin{theorem}
Assume
\begin{itemize}
\item[(H1)] $f(x)$ is conti. in $[a, b]$
\item[(H2)] $f(x)$ is differentiable in $(a, b)$
\end{itemize}
Then $\exists c \in (a, b)$ s.t. $\displaystyle f'(c) = \frac{f(a)-f(b)}{a-b}$
\end{theorem}
\begin{remark}
When $f(a) = f(b)$ in M.V.T, M.V.T becomes Rolle's Thm
\end{remark}
\begin{proof} Mean Value Theorem\\
Let $\displaystyle h(x) = f(a) + \frac{f(a) -f(b)}{a-b} (x-a)$\\
Let $g(x) = f(x) - h(x)$\\
$$\left\{ \begin{array}{rcl}
g(a) & = & f(a) - h(a) = 0\\
g(b) & = & f(b) - h(b) = 0
\end{array}
\right.
$$
$$\displaystyle \frac{h(x) - f(a)}{x-a} = \frac{f(a) - f(b)}{a-b}$$
$g(x) = f(x) - h(x)$ is conti. on $(a,b)$ and diff. on $(a, b)$\\
By Rolle's Thm, $\exists c \in (a, b)$ s.t. $g'(c) = 0$\\
$$\begin{array}{rcl}
h'(x) & = & \displaystyle \frac{f(a) - f(b)}{a-b}\\
g'(x) & = & f'(x) - h'(x)\\
g'(c) & = & \displaystyle f'(c) - \frac{f(a) - f(b)}{a-b}\\
f'(c) & = & \displaystyle \frac{f(a) - f(b)}{a-b}
\end{array}$$
\end{proof}
\begin{eg}
$f(x) = \sin \sqrt{x+1}$
\begin{itemize}
\item Find $f'(x)$
$$\begin{array}{rcl}
f(x) & = & \sin (x+1)^\frac{1}{2}\\\\
f'(x) & = & \displaystyle \cos(\sqrt{x+1}) \cdot \frac{1}{2}(x+1)^{-\frac{1}{2}} \cdot 1\\
& = & \displaystyle \frac{\cos \sqrt{x+1}}{2\sqrt{x+1}}
\end{array}$$
\item 
$\displaystyle \lim_{x \to 0} \frac{\sin \sqrt{x+1 - \sin 1}}{x-0} = \lim_{x \to 0} \frac{f(x) -f(0)}{x-0} = f'(0) = \frac{\cos 1}{2}$
\item Prove $\displaystyle \sin \sqrt{x+1} < \frac{1}{2}x + \sin 1$ for $x>0$\\
By M.V.T, $\exists c \in (0, x)$ s.t. $\displaystyle f'(c) = \frac{f(x) - f(0)}{x-0} = \frac{\sin \sqrt{x+1} - \sin 1}{x}$
$$\displaystyle f'(c) = \frac{\cos \sqrt{c+1}}{2\sqrt{c+1}} \leq \frac{1}{2\sqrt{x+1}} < \frac{1}{2\sqrt{0+1}} = \frac{1}{2}$$
\end{itemize}
\end{eg}
\begin{jk}
你想不到8
\end{jk}

\section{L'Hospital's Rule}
\begin{theorem}
Consider $\displaystyle \lim_{x\to a} \frac{f(x)}{g(x)}$
\begin{itemize}
\item type $\displaystyle \frac{0}{0}$\\
$\displaystyle \lim_{x\to a} f(x) = \lim_{x \to a} g(x) = 0$
\item type $\displaystyle \frac{\infty}{\infty}$\\
$\displaystyle \lim_{x\to a} f(x) = \pm \infty$\\
$\displaystyle \lim_{x\to a} g(x) = \pm \infty$
\end{itemize}
Assume 
\begin{itemize}
\item[(1)] $f$ and $g$ are differentiable
\item[(2)] $g'(x) \neq 0$ on an open interval containing $a$ (except possibility at $a$.)
\end{itemize}
Then for type $\displaystyle \frac{0}{0}$ or $\displaystyle \frac{\infty}{\infty}$, 
$$\displaystyle \lim_{x\to a} \frac{f(x)}{g(x)} = \lim_{x \to a} \frac{f'(x)}{g'(x)}$$
\end{theorem}

\subsection*{Baby L'Hospital's Rule}
\begin{theorem}
Assume 
\begin{itemize}
\item[(1)] $f(a) = g(a) =0$
\item[(2)] $f'$ and $g'$ are conti.
\item[(3)] $g'(a) \neq 0$
\end{itemize}
Then
$$\displaystyle \lim_{x\to a} \frac{f(x)}{g(x)} = \lim_{x\to a} \frac{f'(x)}{g'(x)}$$
\end{theorem}
\begin{proof} Baby L'Hospital's Rule\\
$$\begin{array}{rcl}
\displaystyle \lim_{x\to a} \frac{f(x)}{g(x)} & = & \displaystyle  \lim_{x\to a} \frac{f(x)-f(a)}{g(x)-g(a)}\\
& = & \displaystyle \lim_{x\to a} \frac{\displaystyle \frac{f(x) - f(a)}{x-a}}{\displaystyle \frac{g(x) -g(a)}{x-a}}\\
& = & \displaystyle \frac{\displaystyle \lim_{x\to a} \frac{f(x) -f(a)}{x-a}}{\displaystyle \lim_{x\to a} \frac{g(x) -g(a)}{x-a}} \qquad \qquad \text{\tssteelblue{(limit law)}} \\
& = & \displaystyle \frac{f'(a)}{g'(a)}\\
& = & \displaystyle \frac{\displaystyle \lim_{x\to a} f'(x)}{\displaystyle \lim_{x\to a} g'(x)}\\
& = & \displaystyle \lim_{x\to a} \frac{f'(x)}{g'(x)}
\end{array}$$
\end{proof}
\begin{eg}
$$\begin{array}{ll}
\displaystyle \lim_{x\to 1} \frac{ln x}{x-1} & (\text{type } \displaystyle \frac{0}{0})\\
= \displaystyle \lim_{x\to 1} \frac{\frac{1}{x}}{1}\\
= 1
\end{array}$$
\end{eg}
\begin{eg}
$$\begin{array}{ll}
\displaystyle \lim_{x\to \infty} \frac{e^x}{x^2} & (\text{type } \displaystyle \frac{\infty}{\infty})\\
\displaystyle = \lim_{x\to \infty} \frac{e^x}{2x} & (\text{type } \displaystyle \frac{\infty}{\infty})\\
\displaystyle = \lim_{x\to \infty} \frac{e^x}{2} & (\text{type } \displaystyle \frac{\infty}{2})\\
= \infty
\end{array}$$
$$e^x >> x^2$$
\end{eg}
\begin{eg}
$$\begin{array}{ll}
\displaystyle \lim_{x\to 0} \frac{\tan x -x}{x^3} & (\text{type } \displaystyle \frac{0}{0})\\
\displaystyle = \lim_{x\to 0} \frac{\sec^2 x -1}{3x^2} & (\text{type } \displaystyle \frac{0}{0})\\
\displaystyle = \lim_{x\to 0} \frac{2\sec^2x \tan x}{6x} & (\text{type } \displaystyle \frac{0}{0})\\
\displaystyle = \frac{1}{3} \lim_{x\to 0} \frac{2\sec x \sec x \tan x+ \sec^4 x}{1} & (\text{type } \displaystyle \frac{1}{1})\\
\displaystyle = \frac{1}{3}
\end{array}$$
\end{eg}
\begin{eg}
$$\begin{array}{ll}
\displaystyle \lim_{x\to 0^+} x ln x & (\text{type } 0 \infty)\\
\displaystyle = \lim_{x \to 0^+} \frac{lnx}{\frac{1}{x}}\\
\displaystyle = \lim_{x\to 0^+} \frac{\frac{1}{x}}{-\frac{1}{x^2}}\\
\displaystyle = \lim_{x \to 0^+} (-x)\\
= 0
\end{array}$$
\end{eg}
\begin{eg}
$$\begin{array}{ll}
displaystyle \lim_{x \to \frac{\pi}{2}^-}(\sec x - \tan x) & (\text{type } \infty - \infty)\\
\displaystyle = \lim_{x \to \frac{\pi}{2}^-}(\frac{1}{\cos x} - \frac{\sin x}{\cos x})\\
\displaystyle = \lim_{x \to \frac{\pi}{2}^-} \frac{1-\sin x}{\cos x} & (\text{type } \displaystyle \frac{0}{0})\\
\displaystyle = \lim_{x \to \frac{\pi}{2}^-} \frac{-\cos x}{-\sin x} & (\text{type } \displaystyle \frac{0}{-1})\\
=0
\end{array}$$
\end{eg}
\begin{eg}
$$\lim_{x\to 0^+} (1+ \sin (4x))^{\cot x} \quad (\text{type } f(x)^{g(x)})$$
$$\begin{array}{rcl}
y(x) & = & (1+ \sin (4x))^{\cot x}\\
ln y(x) & = & \cot x \cdot ln(1+\sin (4x))
\end{array}$$
Find $\displaystyle \lim_{x\to 0^+} (ln y(x))$\\
$$\begin{array}{rcll}
\displaystyle \lim_{x\to 0^+} (ln y(x)) & = & \displaystyle \lim_{x\to 0^+} \cot x ln(1+\sin (4x)) & (\text{type }\infty 0)\\
& = & \displaystyle \lim_{x\to 0^+} \frac{ln(1+\sin (4x))}{\tan x} & (\text{ type } \displaystyle \frac{0}{0})\\
& = & \displaystyle \lim_{x\to 0^+} \frac{\frac{4\cos (4x)}{1+\sin (4x)}}{\sec ^2 x} & (\text{type } \frac{\frac{4}{1+0}}{1})\\
& = & 4
\end{array}$$
$$\begin{array}{rcl}
\displaystyle \lim_{x\to 0^+} (ln y(x)) & = & 4\\
\displaystyle ln(\lim_{x\to 0^+} y(x)) & = & \displaystyle \lim_{x\to 0^+} y(x)= e^4)
\end{array}$$
\end{eg}
\begin{jk}
導遊口才很好真不是guide
\end{jk}
\section{Inflection Point (反曲點)}
\begin{defn}
A point $c$ is called a inflection point of $f$ if on $c$, the curve $y=f(x)$ changes from convex ($f''>0,  \ \ \cup$ ) to concave ($f''<0,\ \ \cap$) or concave to convex. That is, if $c$ is an inflection point then $f''(c) = 0$.
\end{defn}
\begin{itemize}
\item If $f' >0 \implies f \nearrow$\\
$f' < 0 \implies f \swarrow$\\
$\nearrow \text{then} \searrow \implies \cap$ \quad (concave downward 凹口向下)\\
$\searrow \text{then} \nearrow \implies \cup$ \quad (concave upward 凹口向上/凸口向下)
\item If $f'' > 0 \implies (f')' > 0 \quad \cup$\\
$f'' < 0 \implies (f')' < 0 \quad \cap$
\end{itemize}
\subsection*{1st derivative test}
$f'(c) = 0$\\
$f'(c^-): +,  \ \ f'(c^+): -, \ \ f'(c) = 0: \text{local max} \quad \cap$\\
$f'(c^-): -, \ \ f'(c^+): +, \ \ f'(c) = 0: \text{local min} \quad \cup$
\subsection*{2nd derivative test}
$f'(c)= 0$\\
$f''(c): +, \ \ f'(c)= 0: \text{local min} \quad \cup$\\
$f''(c): -, \ \ f'(c) = 0: \text{local max} \quad \cap$
\section{Optimization Problems}
$\left \{ \begin{array}{ll}
\text{objective fcn}\\
\text{condition(s)}
\end{array}
\right.$
$\implies \text{max or min}$\\\\
\begin{eg}
Find the point on the parabola $y^2 = 2x$ which closest to $(1, 4)$
$$\left \{ \begin{array}{rcl}
d(x, y) & = & \sqrt{(x-1)^2 + (y-4)^2}\\
x & = & \displaystyle \frac{y^2}{2}
\end{array}\right.$$
Q: Minimize $d(x, y)$ under the condition $y^2= 2x$\\
let
$$\begin{array}{rcl}
f(x, y) & = & (x-1)^2 + (y-4)^2\\
& = & \displaystyle (\frac{1}{2}y^2 -1)^2 + (y-4)^2\\
& = & \displaystyle \frac{1}{4}y^4- \cancel{y^2} + 1 +\cancel{ y^2} - 8y + 16\\
& = & \displaystyle \frac{1}{4}y^4 - 8y+ 17\\
& = & g(y)
\end{array}$$
Find critical point of $g(y)$\\
let $g'(y) = 0$\\
$$g'(y) = y^3-8=0$$
$$(y-2)(y^2+2y+4)=0$$
$$y=2$$
\begin{itemize}
\item 1st derivative test\\
$$\begin{array}{cccc}
f'(c^-) & f'(c) & f'(c^+) & \\
- & 0 & + & \cup \quad \text{local min}\\
+ & 0 & - & \cap \quad \text{local max}
\end{array}$$
$$\begin{array}{rcl}
g'(y) & = & y^3 - 8\\
g'(2) & = & 0\\
g'(2^-) & < & 0\\ 
g'(2^+) & > & 0
\end{array}$$
$\implies \text{local min}$
\item 2nd derivative test\\
$$\begin{array}{ccc}
f'(c) & f''(c) & \\
0 & + & \cup \quad \text{local min}\\
0 & -& \cap \quad \text{local max}
\end{array}$$
$$\begin{array}{rcl}
g''(y) & = & 3y^2\\
g'(2) & = & 12>0
\end{array}$$
$\implies \text{local min}$
\end{itemize}
\end{eg}
\begin{eg}
Find the area of the largest rectangle that can be inscribed (內接) in a semi-circle of radius $r$\\
$$\left \{ \begin{array}{rcl}
A(x, y) & = & 2xy\\
r^2 & = & x^2 + y^2
\end{array}\right.$$
Q: Max $A(x,y)$\\
$$A(x,y) = 2xy = 2x(r^2 - x^2)^{\frac{1}{2}} := f(x)$$
$$\displaystyle f'(x) = 2(r^2 - x^2)^{\frac{1}{2}} + x(-x)(r^2-x^2)^{\frac{1}{2}} = \frac{2}{(r^2 - x^2)^{\frac{1}{2}}}((r^2-x^2)-x^2)$$
critical point\\
let $f'(x) = 0 \implies r^2-2x^2 \implies x = \displaystyle  \frac{r}{\sqrt{2}}$\\
when $f'(x)$ doesn't exist $\implies x=r \implies y=0$
\begin{itemize}
\item 1st derivative test
$$\begin{array}{rcl}
f'(x) & = & \displaystyle \frac{2(r^2-x^2)}{(r^2 - x^2)^{\frac{1}{2}}}\\ % \implies f'(\frac{r}{\sqrt{2}})
\displaystyle f'(\frac{r}{\sqrt{2}}^-) & > & 0\\
\displaystyle f'(\frac{r}{\sqrt{2}} ^+) & < & 0
\end{array}$$
$\implies \text{local max}$
\end{itemize}
\end{eg}
\section{Antiderivatives(反導函數)}
\begin{defn}
$F(x)$ is ``an" antiderivative of $f(x)$ if $F'(x) = f(x)$
\end{defn}
\begin{notn}
Why``an" not ``the"?\\\\
If $F'(x) = f(x)$\\
$\implies (F(x) + c)' = f(x)$ \quad $c$: const. interp of $x$\\
$\implies F(x) + c$ is also an antiderivative of $f(x)$
\end{notn}
\begin{eg}
$$\begin{array}{rcl}
f(x) & = & \sin x\\
F(x) & = & -\cos x + c
\end{array}$$
\end{eg}
\begin{eg}
$$\begin{array}{rcl}
f(x) & = & \displaystyle \frac{1}{x}\\
F(x) & = & ln x + c
\end{array}$$
\end{eg}
\begin{eg}
$$\begin{array}{rcl}
f(x) & = & x^n\\
F(x) & = & \displaystyle \frac{1}{n+1} x^{n+1} + c
\end{array}$$
\end{eg}