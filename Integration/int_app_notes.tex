%
% Copyright 2018 Joel Feldman, Andrew Rechnitzer and Elyse Yeager.
% This work is licensed under a Creative Commons Attribution-NonCommercial-ShareAlike 4.0 International License.
% https://creativecommons.org/licenses/by-nc-sa/4.0/
%

\graphicspath{{./figures/applications/}}

\chapter{Applications of Integration} \label{chap int app}
\section{1st-order linear ODE 一階線性常微分方程}
\begin{defn}[Separable Equations]
\[\begin{array}{rcl}
\displaystyle \frac{dy}{dx} & = & f(x) g(y)\\
\displaystyle \int \frac{dy}{g(y)} & = & \displaystyle \int f(x)dx
\end{array}\]
\end{defn}
\[y'(x) + P(x)y(x) = Q(x) \quad (y \neq 0) \quad \text{where } P(x) \text{ and } Q(x) \text{ are given} \quad ---(\star)\]
\underline{Goal}: solve \(y(x) \)\\
\underline{Idea}: Integrating factor\\
\[(\star) \cdot I(x)\]
\[I(x)y'(x) + I(x)P(x)y(x) = I(x)Q(x)\]
\underline{Hope}: \(Iy' + IPy = (Iy)' \quad \quad ---(1)\)\\
want \( I(x) \) s.t. \((1)\) is true\\
\[(1) \implies Iy' + IPy = I'y + Iy' \quad \text{\tssteelblue{product rule}}\]
\[IP = I' = \displaystyle \frac{dI}{dx}\]
\[\displaystyle \frac{dI}{dx} = I(x)P(x)\]
\[\displaystyle \int \frac{dI}{I} = \int P(x) dx\]
\[\displaystyle ln I \cancel{+ c} = \int P(x) dx\]
\[I(x) \cancel{\cdot e^c} = e^{\int P(x)dx}\]
\[I(x) = \cancel{e^{-c}} e^{\int P(x)dx}\]
i.e: \(I(x) = e^{\int P(x)dx} \quad \text{Integrating factor}\)
\[?= Iy' + IPy = (Iy)'\]
\[e^{\int P(x) dx} y' + e^{\int P(x)dx} Py = (e^{\int P(x)dx} y)'\]
\[\displaystyle \frac{d}{dx}(e^{\int P(x)dx} y) = \frac{d}{dx}(e^{\int P(x)dx})y + e^{\int P(x)dx} y'\]
\[\displaystyle \frac{d}{dx}(e^{\int P(x)dx}) = e^{\int P(x)dx} \quad \text{\tssteelblue{chain rule}}\]
\[\displaystyle \frac{d}{dx} (\int P(x)dx) = P(x) \quad \text{\tssteelblue{F.T.C}}\]
\((1) \implies (Iy)' = IQ\), where \(I(x) = e^{\int P(x)dx}\)
\[\displaystyle \frac{d(Iy)}{dx} = I(x)Q(x)\]
\[\displaystyle \int d(Iy) = \int I(x)Q(x)dx\]
\[\displaystyle Iy \cancel{+ c} = \int I(x)Q(x)dx\]
\[y(x) = \displaystyle \frac{1}{e^{\int P(x)dx}} \int (e^{\int P(x)dx})Q(x)dx\]
\begin{eg}
Solve \(y' + 3x^2y = 6x^2 \quad \quad ---(2)\)

\soln Integrating factor
\[\begin{array}{rcl}
I(x) & = & e^{\int P(x)dx}\\
& = & e^{\int 3x^2dx}\\
& = & e^{x^3 + c}\\
& = & \cancel{e^c} \cdot e^{x^3}
\end{array}\]
\[(2) \cdot e^{x^3} \implies e^{x^3}y' + 3x^2ye^{x^3} = 6x^2e^{x^3}\]
\[\displaystyle \frac{d(e^{x^3})}{dx} = e^{x^3}y' = 6x^2e^{x^3}\]
\[\begin{array}{rcl}
\displaystyle \int d(e^{x^3}y) & = & \displaystyle \int 6x^2e^{x^3}dx\\
& = & \int 2e^{x^3} dx^3\\
& = & 2e^y+c\\
& = & 2e^{x^3} +c
\end{array}\]
\[e^{x^3} y = 2e^{x^3} + c\]
\[y = \displaystyle 2 + \frac{c}{e^{x^3}}\]
\end{eg}
\begin{eg}
Solve Initial value problem (I.V.P) \(\left\{ \begin{array}{rcl}
x^2y' + xy & = & 1 \quad \text{\tssteelblue{ODE}}\\
y(1) & = & 2 \quad \text{\tssteelblue{Initial condition}}
\end{array} \right.\)

\soln 
\[\displaystyle y' + \frac{1}{x}y = \frac{1}{x^2} \quad \quad ---(3)\]
\[I(x) = e^{\int \frac{1}{x}dx} = e^{ln x} = x\]
\[(3) \cdot I(x) \implies xy' + y = \frac{1}{x}\]
\[(xy)' = \frac{1}{x}\]
\[\displaystyle \frac{d(xy)}{dx} = \frac{1}{x}\]
\[\displaystyle \int d(xy) = \int \frac{1}{x} dx\]
\[xy = ln x +c\]
use \(y(1) = 2\) to determine \(c\)
\[xy = ln x +c\]
\[x=1, y=2\]
\[1 \cdot 2 = ln 1 +c\]
\[c = 2\]
The solution is \(xy = ln x + 2 \text{ or } y = \displaystyle \frac{lnx}{x} + \frac{2}{x}\)
\end{eg}
\begin{eg}
Solve \((\sec x)y' - y = \tan x e^{\cos x - \sin x} \quad \quad ---(4)\)

\soln
\[\displaystyle y' - \frac{y}{\sec x} = \sin x e^{\cos x - \sin x}\]
\[I(x) = e^{\int \cos x dx} = e^{\sin x}\]
\[\begin{array}{rcl}
(4) \cdot I(x) \implies e^{\sin x}y' + \cos x e^{\sin x}y & = & \cancel{e^{\sin x}} \sin x e^{\cos x - \cancel{\sin x}}\\
& = & \sin x e^{\cos x}
\end{array}\]
\[(e^{\sin x}y)' = \sin x e^{\cos x}\]
\[\displaystyle \frac{d(e^{\sin x}y)}{dx} = \sin x e^{\cos x}\]
\[\int d(e^{\sin x}y) = \int \sin x e^{\cos x}dx\]
\[\begin{array}{rcl}
e^{\sin x}y & = & \displaystyle -\int e^{\cos x}d\cos x\\
& = & -e^y + c\\
& = & -e^{\cos x} + c
\end{array}\]
\[e^{\sin x}y = -e^{\cos x} + c\]
\(y' + P(x)y = Q(x) \implies \text{ IF is } I(x) = e^{\int P(x)dx}\)
\end{eg}
\section{Arc Length  弧長}
\begin{defn}
\[ds = \sqrt{(dx)^2 + (dy)^2}\]
\[\begin{array}{rcl}
\displaystyle \int^{x=b}_{x=a} ds & = &\displaystyle \int^{x = b}_{x = a} \sqrt{(dx)^2 + (dy)^2}\\
& = & \displaystyle \int^{x = b}_{x = a} \sqrt{\frac{dx^2}{dx^2} + \frac{dy^2}{dx^2}} dx\\
& = & \displaystyle \int^b_a \sqrt{1 + (f'(x))^2} dx\\
\end{array}\]
Given \(x = g(y), c \leq y \leq d\), arc length \( = \displaystyle \int^d_c \sqrt{1 + (g'(y))^2} dy\)
\end{defn}
\begin{eg}
Find the arc length of \(y = x^{\frac{3}{2}} \text{ from } (1, 1) \text{ to } (4, 8)\)

\soln
\[\begin{array}{rcl}
f(x) = x^{\frac{3}{2}} & \implies & \displaystyle f'(x) = \frac{2}{3} x^{\frac{1}{2}}\\
& \implies & \displaystyle 1 + (f'(x))^2 = 1 + \frac{4}{9} x\\
\end{array}\]
\[\begin{array}{rcl}
\displaystyle \int \sqrt{1 + x}dx & = & \displaystyle \int (1 + x)^{\frac{1}{2}} dx\\
& = & \displaystyle \frac{2}{3} (1 + x)^{\frac{3}{2}} + c
\end{array}\]
\[\begin{array}{rcl}
\displaystyle \int^4_1 \sqrt{1 + \frac{4}{9} x} dx & = & \displaystyle \frac{4}{9} \frac{2}{3} (1 + \frac{4}{9} x)^{\frac{3}{2}}\Big|^4_1\\
& = & \displaystyle \frac{8}{27} (10^{\frac{3}{2}} - \frac{13^{\frac{3}{2}}}{8})\\
& = & \displaystyle \frac{1}{27} (40^{\frac{3}{2}} - 13^{\frac{3}{2}})
\end{array}\]
\end{eg}
\begin{eg}
Find the arc length of \(y = e^x \text{ from } (ln3, 3) \text{ to } (ln8, 8)\)

\soln
\[\begin{array}{rcl}
f(x) = e^x & \implies & f'(x) = e^x\\
& \implies & (f'(x))^2 = e^{2x}
\end{array}\]
\[\begin{array}{rcl}
I & = & \displaystyle \int \sqrt{1 + e^{2x}}dx\\
\text{let } u & = & \sqrt{1 + e^{2x}} = (1 + e^{2x})^{\frac{1}{2}}\\
du & = & \displaystyle \cancel{\frac{1}{2}} (1 + e^{2x})^{-\frac{1}{2}} \cancel{2} e^{2x} dx = \frac{u^2 - 1}{u} dx\\
I & = & \displaystyle \int u (\frac{u}{u^2 - 1}du)\\
& = & \displaystyle \int \frac{u^2}{u^2 - 1} du\\
& = & \displaystyle \int (1 + \frac{1}{u^2 - 1})du\\
& = & \displaystyle u + \frac{1}{2}(ln(u - 1) - ln(u + 1)) + c
\end{array}\]
\[\begin{array}{rcl}
\displaystyle \int^{ln8}_{ln3} \sqrt{1 + e^{2x}}dx & = & \displaystyle \sqrt{1 + e^{2x}} + \frac{1}{2}(ln( \sqrt{1 + e^{2x}} -1) - ln(\sqrt{1 + e^{2x}} + 1))\Big|^{ln8}_{ln3}\\
& = & \displaystyle \sqrt{1 + 64} + \frac{1}{2} (ln(\sqrt{65} - 1) - ln(\sqrt{65} + 1))\\
&& \displaystyle - (\sqrt{1 + 9} + \frac{1}{2} (ln(\sqrt{10} - 1) - ln(\sqrt{10} + 1)))\\
& = & \displaystyle \sqrt{65} + \frac{1}{2}ln(\frac{\sqrt{65} - 1}{\sqrt{65} + 1}) - (\sqrt{10} + \frac{1}{2} ln(\frac{\sqrt{10} - 1}{\sqrt{10} + 1}))\\
& = & \displaystyle \sqrt{65} - \sqrt{10} + \frac{1}{2}(ln (\frac{(\sqrt{65} - 1)^2}{65 -1}) - ln (\frac{(\sqrt{10} - 1)^2}{10 - 1}))\\
& = & \displaystyle \sqrt{65} - \sqrt{10} + \frac{1}{2} (ln (\frac{(\sqrt{65} - 1)^2}{(\sqrt{10} -1)^2}) + ln (\frac{9}{64}))\\
& = & \displaystyle \sqrt{65} - \sqrt{10} + ln(\frac{\sqrt{65} - 1}{\sqrt{10} - 1}) + ln \frac{3}{8}\\
& = & 2 + ln3 - ln2
\end{array}\]
\end{eg}
\section{Calculus with Parametric Curve}
\begin{defn}[Parametric Equations 參數式]
\[y = f(x)\]
\[\left\{ \begin{array}{rcl}
y & = & y(t)\\
x & = & x(t)
\end{array} \right. \quad \quad t : \text{parameter}\]
\end{defn}
\begin{notn}
\[\left\{ \begin{array}{rcl}
x & = & f(t)\\
y & = & g(t)
\end{array} \right. \quad \quad t : \text{parameter}\]
\begin{itemize}
\item Tangent
\[\text{slope} \displaystyle = \frac{dy}{dx} = \frac{\frac{dy}{dt}}{\frac{dx}{dt}} = \frac{g'(t)}{f'(t)}\]
\[\displaystyle g'(t) = \frac{dy}{dt} = \frac{dy}{dx} \frac{dx}{dt} = \frac{dy}{dx}f'(t) \quad \text{\tssteelblue{chain rule}}\]
\item Area
\[\displaystyle \int^b_a y(x)dx = \int^{t_2}_{t_1} y(f(t)) \frac{dx}{dt} dt = \int^{t_2}_{t_1} y(f(t))f'(t)dt\]
\item Arc Length
\[\displaystyle \int ds = \int \sqrt{(dx)^2 +(dy)^2} = \int \sqrt{\frac{(dx)^2}{(dt)^2} + \frac{(dy)^2}{(dt)^2}} dt = \int \sqrt{(f'(t))^2 + (g'(t))^2}dt\]
\item Surface Area\\
\(y = y(x)\) around \(x\)-axis
\[\displaystyle \int 2 \pi y(x)ds = \int 2 \pi y(f(t)) \cdot \sqrt{(f'(t))^2 + (g'(t))^2} dt\]
\item Cycloid 擺線
\[\overset\frown{PQ} = \overline{P_0 Q}\]
\[x = \overline{P_0Q} - r \sin \theta = \overset\frown{PQ} - r \sin \theta = r \theta - r \sin \theta\]
\[y = r - r \cos \theta\]
\[\left\{\begin{array}{rcl}
x(\theta) & = & r(\theta - \sin \theta)\\
y(\theta) & = & r(1 - \cos \theta)
\end{array}\right.\]
\end{itemize}
\end{notn}
\begin{eg}
\[\left\{ \begin{array}{rcl}
x & = & \cos t\\
y & = & \sin t
\end{array} \right.\] 
\[\implies x^2 + y^2 = \cos^2 t + \sin^2 t = 1\]
circle centered at \((0, 0)\) with radius \(1\)
\end{eg}
\begin{eg}
Find the slope of the tangent of Cycloid at \(\displaystyle \theta = \frac{\pi}{3}\)

\soln
\[\begin{array}{rcl}
x(\theta) & = & r(\theta - \sin \theta)\\
y(\theta) & = & r(1 - \cos \theta)
\end{array}\]
\[\begin{array}{rcl}
\text{slope} & = & \displaystyle \frac{y'(\theta)}{x'(\theta)} \Big |_{\theta = \frac{\pi}{3}}\\
& = & \displaystyle \frac{r \sin \theta}{r(1 - \cos \theta)} \Big |_{\theta = \frac{\pi}{3}}\\
& = & \displaystyle \frac{\frac{\sqrt{3}}{2}}{1 - \frac{1}{2}}\\
& = & \sqrt{3}
\end{array}\]
\end{eg}
\begin{eg}
\[\begin{array}{rcl}
\text{area } A & = & \displaystyle \int^{2\pi}_{\theta} r(1 - \cos \theta) \cdot r(1 - \cos \theta)d\theta\\
& = & \displaystyle r^2 \int^{2\pi}_{\theta} (1 - \cos \theta)^2 d\theta\\
& = & \displaystyle r^2 (\theta - 2 \sin \theta + \frac{1}{4} \sin 2\theta + \frac{1}{2} \theta) \Big |^{2\pi}_0\\
& = & \displaystyle r^2 (\frac{3}{\cancel{2}} \cdot \cancel{2}\pi)\\
& = & 3 \pi r^2
\end{array}\]
\[\cos 2 \theta = 2 \cos^2 \theta -1\]
\[\begin{array}{rcl}
\displaystyle \int (1 - 2 \cos \theta + \cos^2 \theta) d\theta & = & \displaystyle \int \cos^2 \theta d\theta\\
& = & \displaystyle \int \frac{1}{2} (\cos 2 \theta+1) d\theta
\end{array}\]
\[\begin{array}{rcl}
\text{arc length } & = & \displaystyle \int^{2 \pi}_0 r \sqrt{(1 - \cos \theta)^2 + (\sin \theta)^2} d \theta\\
& = & \displaystyle r \int^{2\pi}_0 \sqrt{2 - 2\cos \theta} d\theta\\
& = & \displaystyle \sqrt{2}r \int^{2\pi}_0 \sqrt{1 - \cos \theta} d\theta\\
& = & \displaystyle \sqrt{2}r \int^{2\pi}_0 \sqrt{2\sin^2 \frac{\theta}{2}} d\theta\\
& = & \displaystyle 2r \int^{2\pi}_0 \sqrt{\sin^2 \frac{\theta}{2}} d\theta\\
& = & \displaystyle 2r \int^{2\pi}_0 \sin \frac{\theta}{2} d\theta\\
& = & \displaystyle 2r(-2\cos \frac{\theta}{2}) \Big|^{2\pi}_0\\
& = & -4r((-1)-1)\\
& = & 8r
\end{array}\]
\[\begin{array}{rcl}
\cos 2\theta = 2\cos^2 \theta - 1 & = & 1 - 2 \sin^2 \theta\\
1 - \cos 2\theta & = & 2 \sin^2 \theta\\
1 - \cos \theta & = & \displaystyle 2 \sin^2 \frac{\theta}{2}
\end{array}\]
\[\begin{array}{rcl}
\text{surface area } & = & \displaystyle \int 2\pi (r(1 - \cos \theta)(\sqrt{2}r \sqrt{1 - \cos \theta}) d\theta\\
& = & \displaystyle 2 \sqrt{2} \pi r^2 \int (1 - \cos \theta)^{\frac{1}{2}} d\theta
\end{array}\]
\[\begin{array}{rcl}
\displaystyle \int \sin^3 \theta d\theta & = & \displaystyle \int \sin^2 \theta \sin \theta d\theta\\
& = & \displaystyle - \int (1 - z^2) dz
\end{array}\]
\end{eg}
\begin{eg}
Find the surface area generated by rotating w.r.t \(x\)-axis

\soln
\[\begin{array}{rcl}
A & = & \displaystyle \int^{2\pi}_0 2\pi y(\theta) ds\\
& = & \displaystyle 2\pi \sqrt{2} r^2 \int^{2\pi}_0 (1 - \cos \theta) \sqrt{1 - \cos \theta} d\theta\\
& = & \displaystyle 2\pi \sqrt{2} r^2 \int^{2\pi}_0 2 \sin^2 \frac{\theta}{2} \sqrt{2} \sqrt{\sin^2 \frac{\theta}{2}} d\theta\\
& = & \displaystyle 8\pi r^2 \int^{2\pi}_0 \sin^2 \frac{\theta}{2} \sin \frac{\theta}{2} d\theta\\
& = & \displaystyle -16\pi r^2 \int^{2\pi}_0 (1 - \cos^2 \frac{\theta}{2}) d\cos \frac{\theta}{2}\\
& = & \displaystyle -16\pi r^2 \int^{-1}_1 (1 - z^2) dz\\
& = & \displaystyle -16\pi r^2 (z - \frac{1}{3} z^3) \Big|^{-1}_1\\
& = & \displaystyle -16\pi r^2 (-1 + \frac{1}{3} - (1 - \frac{1}{3}))\\
& = & \displaystyle -16\pi r^2 (-\frac{4}{3})\\
& = & \displaystyle \frac{64}{3} \pi r^2
\end{array} \quad \quad \begin{array}{rcl}
ds & = & \displaystyle \sqrt{(dx)^2 + (dy)^2}\\
& = & \displaystyle \sqrt{(\frac{dx}{d\theta})^2 + (\frac{dy^2}{d\theta})^2}\\
& = & \displaystyle \sqrt{2} r \sqrt{1 - \cos \theta} d\theta
\end{array}\]
\end{eg}
\section{Polar Coordinates 極坐標}
\begin{defn}
\[\left\{\begin{array}{rcl}
x & = & r \cos \theta \quad ---(1)\\
y & = & r \sin \theta \quad ---(2)
\end{array}\right.\]
\[\begin{array}{rcl}
\displaystyle \frac{(2)}{(1)} = \frac{y}{x} = \tan \theta & \implies & \displaystyle \theta = \tan^{-1} \frac{y}{x}\\
(1)^2 + (2)^2 = r^2 + 1 & \implies & r = \sqrt{x^2 + y^2}
\end{array}\]
\[(r, \theta) = (\sqrt{x^2 + y^2}, \tan^{-1} \frac{y}{x})\]
\end{defn}
\begin{notn}
\[\displaystyle (1, \frac{\pi}{4}) = (1, \frac{\pi}{4} + 2\pi) = (1, \frac{9}{4}\pi) = (-1, \frac{5}{4}\pi)\]
Expression of the same point by polar coordinate may not be unique.
\end{notn}
\subsection*{Symmetry 對稱}
\begin{itemize}
\item \(x\)-axis\\
\(\theta \to -\theta\), the eqn is invariant 
\item \(y\)-axis\\
\(\theta \to \pi - \theta\), the eqn is invariant 
\item origin\\
\( r \to -r\), the eqn is invariant\\\\
\end{itemize}
\begin{eg}
Plot the graph of \(r = f(\theta) = 2 \cos(2\theta)\)
\begin{itemize}
\item \(\theta \to - \theta\)
\[r = 2 \cos(2\theta) = 2 \cos(2(-\theta))\]
\(\therefore\) the graph is symmetric w.r.t. \(x\)-axis
\item \(\theta \to \pi - \theta\)
\[r = 2 \cos(2\theta) = 2 \cos(2 (\pi - \theta))\]
\[\cos(2\pi - 2\theta) = \cos (2\pi) \cos(2\theta) + \sin (2\pi) \sin(2\theta) = \cos (2\theta)\]
\(\therefore\) the graph is symmetric w.r.t. \(y\)-axis
\item \(r \to -r\)
\[-r = 2 \cos (2\theta)  \quad \quad r = -2 \cos (2\theta)\]
\[\theta \to \theta + \pi\]
\[r = 2\cos (2\theta) = 2 \cos(2(\theta + \pi)) = 2\cos(2\theta) \cos(2\pi) - \sin(2\theta) \sin(2\pi) = 2 \cos (2\theta)\]
\(\therefore\) the graph is symmetric w.r.t. to \((0, 0)\)
\begin{center}
\begin{tabular}{c|c|c|c|c|c}
\(\theta\) & \(0\) & \(\displaystyle \frac{\pi}{6}\) & \(\displaystyle \frac{\pi}{4}\) & \(\displaystyle \frac{\pi}{3}\) & \(\displaystyle \frac{\pi}{2}\)\\ \hline
\(r\) & \(2\) & \(\sqrt{3}\) & \(0\) & \(-1\) & \(-2\)
\end{tabular}
\end{center}
\end{itemize}
\end{eg}
\begin{eg}
Find tangent of \(\displaystyle r = 2\cos (2\theta) \text{ at } (1, \frac{\pi}{6})\)

\soln
\[\left\{\begin{array}{rcl}
y & = & r \sin \theta = 2\cos (2\theta) \cdot \sin \theta = y(\theta)\\
x & = & r \cos \theta = 2\cos (2\theta) \cdot \cos \theta = x(\theta)
\end{array}\right.\]
\[\displaystyle \frac{dy}{dx} \Big|_{\theta = \frac{\pi}{6}} = \frac{\frac{dy}{d\theta}}{\frac{dx}{d\theta}} \Big|_{\theta = \frac{\pi}{6}} = \frac{\cancel{2}(-2\sin (2\theta) \sin \theta + \cos(2\theta)) \cdot \cos \theta}{\cancel{2} (-2\sin (2\theta) \cos \theta + \cos(2\theta)) \cdot (-\sin \theta)}\]
\end{eg}
\begin{eg}[Cardioid 心臟線]
Plot the graph of \(r = 1 + \sin \theta = f(\theta)\)
\begin{itemize}
\item \(\theta \to -\theta\)
\[r = 1 + \sin (-\theta) = 1 - \sin \theta\]
\(\therefore\) the graph is NOT symmetric w.r.t. \(x\)-axis 
\item \(\theta \to \pi - \theta\)
\[r = 1 + \sin (\pi - \theta) = 1 + \sin\pi \cos \theta - \cos\pi \sin \theta = 1 + \sin\theta\]
\(\therefore\) the graph is symmetric w.r.t. \(y\)-axis
\begin{center}
\begin{tabular}{c|c|c|c|c|c}
\(\theta\) & \(\displaystyle - \frac{\pi}{2}\) & \(\displaystyle - \frac{\pi}{6}\) & \(o\) & \(\displaystyle \frac{\pi}{2}\) & \(\displaystyle \frac{\pi}{2}\)\\ \hline
\(r\) & \(0\) & \(\displaystyle \frac{1}{2}\) & \(1\) & \(\displaystyle \frac{3}{2}\) & \(2\)
\end{tabular}
\end{center}
\end{itemize}
\end{eg}
\begin{eg}
Find the slope of the tangent of \(r = 1 + \sin \theta \text{ at } \theta = \displaystyle \frac{\pi}{3}\)

\soln
\[\displaystyle \frac{dy}{dx} \Big|_{\theta = \frac{\pi}{3}}= \frac{\frac{dy}{d\theta}}{\frac{dx}{d\theta}} \Big|_{\theta \frac{\pi}{3}} = \frac{\cos \theta \sin \theta + (1 + \sin \theta) \cos \theta}{\cos \theta \cos \theta + (1 + \sin \theta) (1 - \sin \theta)} \Big|_{\frac{\pi}{3}}\]
\end{eg}
\subsection*{Area and Arc Length in Polar Coordinates}
\begin{itemize}
\item Area
\[dA = \displaystyle \cancel{\pi} r^2 \frac{d\theta}{2\cancel{\pi}} = \frac{2}{1} r^2 d\theta\]
\[\displaystyle \int A = \frac{1}{2} \int^b_a r^2 d\theta\]
\item Arc length
\[x = f(\theta)\cos \theta\]
\[y = f(\theta) \sin \theta\]
\[\begin{array}{rcl}
\displaystyle (\frac{dx}{d\theta})^2 + (\frac{dy}{d \theta})^2 & = & (f'(\theta) \cos \theta + f(\theta) (-\sin \theta))^2 + (f'(\theta) \sin \theta + f(\theta) \cos \theta)^2\\
& = & (f'(\theta))^2 + (f(\theta))^2
\end{array}\]
\[\begin{array}{rcl}
ds \sqrt{(dx)^2 + (dy)^2} & = & \displaystyle \sqrt{(\frac{dx}{d\theta})^2 + (\frac{dy}{d\theta})^2} d\theta\\
& = & \sqrt{(f'(\theta))^2 + (f(\theta))^2} d\theta\\
& = & \sqrt{(r')^2 + r^2} d\theta
\end{array}\]
\end{itemize}
\begin{eg}
Find the length of \( r = 1 + \sin \theta\)

\soln
\[r = f(\theta) = 1 + \sin \theta\] 
\[f'(\theta) = \cos \theta\]
\[\begin{array}{rcl}
(f(\theta))^2 + (f'(\theta))^2 & = & (1 + \sin \theta)^2 + \cos^2 \theta\\
& = & 2 + 2\sin \theta
\end{array}\]
\[\begin{array}{rcl}
\text{length } & = & \displaystyle 2 \int^{\frac{\pi}{2}}_{- \frac{\pi}{2}} \sqrt{2 + 2\sin \theta} d\theta\\
& = & 2 \sqrt{2} \displaystyle \int^{\frac{\pi}{2}}_{-\frac{\pi}{2}} \sqrt{1 + \sin \theta} d\theta\\
& = & 2 \sqrt{2} \displaystyle \int^{\frac{\pi}{2}}_{-\frac{\pi}{2}} \sqrt{\sin^2 \frac{\theta}{2} + \cos^2 \frac{\theta}{2} + 2 \sin \frac{\theta}{2} \cos \frac{\theta}{2}}\\
& = & 2 \sqrt{2} \displaystyle \int^{\frac{\pi}{2}}_{-\frac{\pi}{2}} \sqrt{(\sin \frac{\theta}{2} + \cos \frac{\theta}{2})^2} d\theta\\
& = & 2 \sqrt{2} \displaystyle \int^{\frac{\pi}{2}}_{-\frac{\pi}{2}} \Big| \sin \frac{\theta}{2} + \cos \frac{\theta}{2} \Big| d\theta\\
& = & 2 \sqrt{2} \displaystyle (\int^{0}_{-\frac{\pi}{2}} \Big| \sin \frac{\theta}{2} + \cos \frac{\theta}{2} \Big| d\theta + \int^{\frac{\pi}{2}}_0 \Big| \sin \frac{\theta}{2} + \cos \frac{\theta}{2} \Big| d\theta)\\
& = & 2 \sqrt{2} \displaystyle \int^{\frac{\pi}{2}}_{-\frac{\pi}{2}} \sin \frac{\theta}{2} + \cos \frac{\theta}{2} d\theta\\
& = & 2 \sqrt{2} \displaystyle (-2 \cos \frac{\theta}{2} + 2 \sin \frac{\theta}{2}) \Big|^{\frac{\pi}{2}}_{-\frac{\pi}{2}}\\
& = & 2 \sqrt{2} \displaystyle (\cancel{-2 \frac{\sqrt{2}}{2}} + \cancel{2 \frac{\sqrt{2}}{2}} - (-2 \frac{\sqrt{2}}{2} - 2 \frac{\sqrt{2}}{2}))\\
& = & 2 \sqrt{2} \cdot 2 \sqrt{2}\\
& = & 8
\end{array}\]
\end{eg}
\begin{eg}[Rose Curve 玫瑰線]
\begin{itemize}
\item Find the area of one loop of \(r = 2 \cos 2 \theta\)
\[\begin{array}{rcl}
A & = & \displaystyle \int \frac{1}{2} r^2 d\theta\\
& = & \displaystyle \cancel{2} \int^{\frac{\pi}{4}}_0 \cancel{\frac{1}{2}} (2 \cos 2 \theta)^2 d\theta\\
& = & \displaystyle 4 \int^{\frac{\pi}{4}}_0 \cos^2 2\theta d\theta\\
& = & \displaystyle 2 \int^{\frac{\pi}{4}}_0 (0 + \cos 4\theta) d\theta\\
& = & \displaystyle 2 (\theta + \frac{1}{4} \sin 4\theta) \Big|^{\frac{\pi}{4}}_0\\
& = & \displaystyle 2 (\frac{\pi}{4})\\
& = & \displaystyle \frac{\pi}{2}
\end{array}\]
\item Find the arc length of one loop of \(r = 2 \cos 2 \theta\)
\end{itemize}
\end{eg}
\begin{eg}[Cardioid]
\begin{itemize}
\item Plot the graph of \(r = 1 - \cos \theta\)
\begin{center}
\begin{tabular}{c|c|c|c}
\(\theta\) & \(0\) & \(\pi\) & \(\displaystyle \frac{\pi}{2}\)\\\hline
\(r\) & \(0\) & \(2\) & \(1\)
\end{tabular}
\end{center}
\item Find the area of \(r = 1 - \cos \theta\)
\[\begin{array}{rcl}
A & = & \displaystyle \cancel{2} \int^{\pi}_0 \cancel{\frac{1}{2}} (1 - \cos \theta)^2 d\theta\\
& = & \displaystyle \int^{\pi}_0 (1 - 2 \cos \theta + \cos^2 \theta) d\theta\\
& = & \displaystyle \frac{3}{2} \theta - 2 \sin \theta + \frac{1}{4} \sin 2 \theta \Big|^{\pi}_0\\
& = & \displaystyle \frac{3}{2} \pi
\end{array}\]
\end{itemize}
\end{eg}