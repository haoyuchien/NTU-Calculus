%
% Copyright 2018 Joel Feldman, Andrew Rechnitzer and Elyse Yeager.
% This work is licensed under a Creative Commons Attribution-NonCommercial-ShareAlike 4.0 International License.
% https://creativecommons.org/licenses/by-nc-sa/4.0/
%
\graphicspath{{figures/series/}}
\chapter{Sequence and Series}\label{chap seq ser}
\section{Sequence 數列}
\begin{defn}[\(f(n), n \in N\)]
\begin{itemize}
\item \(\displaystyle a_n = \frac{1}{n} \quad (\frac{1}{1}, \frac{1}{2}, \frac{1}{3}, \cdots)\)
\item \(\displaystyle a_n = n \quad (1, 2, 3, \cdots)\)
\item \(\displaystyle a_n = (-1)^n \quad (-1, 1, -1, 1, \cdots)\)
\end{itemize}
\end{defn}
Q: Given a sequence (infinite) \(\{a_n\}^{\infty}_{n = 1}\). Is \(\{a_n\}^{\infty}_{n = 1}\) convergent or divergent?
\begin{itemize}
\item \(\displaystyle a_n = \frac{1}{n} \quad (\frac{1}{1}, \frac{1}{2}, \frac{1}{3}, \cdots)\)\\
\(\displaystyle \lim_{n \to \infty} a_n = 0\) \quad convergent 
\item \(\displaystyle a_n = n \quad (1, 2, 3, \cdots)\)\\
\(\displaystyle \lim_{n \to \infty} a_n = \infty\) \quad divergent 
\item \(\displaystyle a_n = (-1)^n \quad (-1, 1, -1, 1, \cdots)\)\\
divergent
\end{itemize}
\begin{theorem}
If \(\displaystyle \lim_{x \to \infty} a_x = L, x \in R\), then
\[\lim_{n \to \infty} a_n = L, n \in N\]
\end{theorem}
\begin{eg}
\[\displaystyle a_n = \frac{ln n}{n}, n \in N\]
\[\displaystyle \lim_{x \to \infty} \frac{ln x}{x}, x \in R = \lim_{x \to \infty} \frac{\frac{1}{x}}{1} = 0\]
\[\displaystyle \lim_{n \to \infty} \frac{ln n}{n} = 0, n \in N\]
\end{eg}
\begin{eg}[Geometric Sequence]
\(\{r^n\}^{\infty}_{n = 1}\) converges
\[\{r^n\}^{\infty}_{n = 1} = \{r, r^2, r^3, \cdots\}\]
\[r = -1 \{-1, 1, -1, 1, \cdots\}\]
\[-1 < r \leq 1\]
\end{eg}
\begin{theorem}
Assume
\begin{itemize}
\item[(1)] \(\{a_n\}^{\infty}_{n = 1}\) is monotone 單調遞增或遞減\\
\(a_n \leq a_{n + 1}, \forall n = 0, 1, 2, \cdots \text{ or } a_n \geq a_{n + 1}, \forall n = 0, 1, 2, \cdots\)
\item[(2)]\(\{a_n\}^{\infty}_{n = 1}\) is bounded 有界\\
\(\Big| a_n \Big| \leq M, \forall n = 1, 2, 3, \cdots \text{ for some } M > 0\)
\end{itemize}
\(\implies \{a_n\}^{\infty}_{n = 1}\) converges
\end{theorem}
\begin{eg}
\[x = \sqrt{2+ \sqrt{2+ \sqrt{2 + \cdots}}}\]
\[\begin{array}{rcl}
x^2 & = & 2 + \sqrt{2 + \sqrt{2 + \cdots}}\\
& = & 2 + x
\end{array}\]
\[x^2 -x - 2 = 0\]
\[(x + 1)(x -2) = 0\]
\[x = -1 \text{ or } 2\]
\[x = 2\]
\[\begin{array}{rclcl}
a_1 & = & \sqrt{2}\\
a_2 & = & \sqrt{2 + \sqrt{2}} & = & \sqrt{2 + a_1}\\
a_3 & = & \sqrt{2 + \sqrt{2 + \sqrt{2}}} & = & \sqrt{2 + a_2}\\
a_n &&& = & \sqrt{2 + a_n}
\end{array}\]
Consider \(\{a_n\}^{\infty}_{n = 1}\) 
\begin{itemize}
\item \(a_n \text{ monotone } (a_n \nearrow \text{ as } n \nearrow)\)
\item \(a_n\) bounded
\end{itemize}
Want to show \(a_n \leq 2, \forall n \in N \quad --- (\star)\)\\
Use Induction (數學歸納法)
\begin{itemize}
\item \(n = 1\)\\
\[\Big| a_1 \Big| = \Big| \sqrt{2} \Big| = \sqrt{2} \leq 2\]
\(\therefore n = 1, (\star)\) is true
\item If \(n = k, (\star)\) is true \(\implies n = k + 1, (\star)\) is true\\
\[\Big| a_{k + 1} \Big|= \Big| \sqrt{2 + a_k} \Big| \leq \sqrt{2 + 2} = 2\]
\[\Big| a_{k + 1} \Big| \leq 2\]
\end{itemize}
By Induction, \((\star)\) is true\\
By Thm, \(\{a_n\}^{\infty}_{n = 1}\) converges 
\end{eg}
\section{Series 級數}
\begin{defn}[Series]
Summation of a sequence
\[\displaystyle \sum^{\infty}_{n = 1} a_n = a_1 + a_2 + a_3 + \cdots\]
\end{defn}
Q: Convergence of a series?
\[\displaystyle \sum^{\infty}_{r = 1} r^n = r + r^2 + r^3 + \cdots = \lim_{n \to \infty} \frac{r(1 - r^n)}{1 - r}\]
\[\text{convergence} \Leftrightarrow - 1 < r < 1\]
\begin{theorem}
If \(\displaystyle \sum^{\infty}_{n = 1} a_n\) converges, then
\[\lim_{n \to \infty} a_n = 0\]
\end{theorem}
\begin{notn}
\(\displaystyle \lim_{n \to \infty} a_n = 0\) does not imply \(\sum^{\infty}_{n = 1} a_n\) converges
\end{notn}
\begin{eg}
\[\displaystyle \sum^{\infty}_{n = 1} \frac{1}{n} = 1 + \frac{1}{2} + \frac{1}{3} + \frac{1}{4} + \cdots\]
Assume \(\displaystyle \sum^{\infty}_{n = 1} \frac{1}{n} = S\) converges
\[\begin{array}{rccccccccl}
S & = & 1 & + & \displaystyle (\frac{1}{2} + \frac{1}{3}) & + & \displaystyle (\frac{1}{4} + \frac{1}{5} + \frac{1}{6} + \frac{1}{7}) & + & \displaystyle (\frac{1}{8} + \frac{1}{9} + \cdots + \frac{1}{15}) & + \cdots\\
&&&& > \frac{1}{2} && > \frac{1}{8} + \frac{1}{8} + \frac{1}{8} + \frac{1}{8} && > \frac{1}{16} + \frac{1}{16} + \cdot + \frac{1}{16}\\
& > & \displaystyle 1 & + & \displaystyle \frac{1}{2} & + & \displaystyle \frac{1}{2} & + & \displaystyle \frac{1}{2} & + \cdots\\
& = & \infty
\end{array}\]
\(\displaystyle \lim_{n \to \infty} a_n = \lim_{n \to \infty} \frac{1}{n} = 0\ \text{ but } \sum^{\infty}_{n = 1} \frac{1}{n}\) diverges
\end{eg}
\section{Integral Test}
\begin{defn}
Assume
\begin{itemize}
\item[(1)] \(a_n > 0\) \quad for \(n = m, m + 1, \cdots\) (essentially positive)
\item[(2)] \(a_n \searrow\)
\item[(3)] \(a_n\) is conti. for \(x \in [1, \infty)\)
\end{itemize}
\(\implies \displaystyle \sum^{\infty}_{n = 1} a_n \text{ and } \int^{\infty}_1 a_x dx\) both converge or diverge
\end{defn}
\begin{eg}
Determine \(\displaystyle \sum^{\infty}_{n = 1} \frac{1}{n^p}, p > 0\) (P-series) converges or diverges\\

\soln
\begin{itemize}
\item[(1)] \(\displaystyle \frac{1}{n^p} > 0 \quad \forall n = 1, 2, 3, \cdots\)
\item[(2)] \(\displaystyle a_n = \frac{1}{n^p} \searrow\)
\item[(3)] \(a_x = \frac{1}{x^p}\) conti for \(x \geq 1\)
\end{itemize}
By integral test \(\implies \displaystyle \sum^{\infty}_{n = 1} \frac{1}{n^p} \text{ and} \int^{\infty}_1 \frac{1}{x^p} dx\) both converges or diverges
\end{eg}
\begin{eg}
Determine \(\displaystyle \sum^{\infty}_{n = 1} \frac{1}{n^2 +1}\) converges or diverges by integral test

\soln
\begin{itemize}
\item[(1)] \(\displaystyle a_n = \frac{1}{n^2 + 1} > 0 \quad \forall n = 1, 2, 3, \cdots\)
\item[(2)] \(\displaystyle a_n =\frac{1}{n^2 + 1} \searrow \quad \quad f(x) = \frac{1}{n^2 + 1} \implies f'(x) = -(x^2 + 1)^{-2}(2x) < 0, x> 0\)
\item[(3)] \(a_x = \frac{1}{x^2 + 1}\) is conti. \(x \geq 1\)
\end{itemize}
By integral test, consider
\[\begin{array}{rcl}
\displaystyle \int^{\infty}_1 \frac{1}{x^2 + 1} dx & = & \displaystyle \lim_{t \to \infty} (\int^{\infty}_1 \frac{1}{x^2 + 1} dx)\\
& = & \displaystyle \lim_{t \to \infty} (\tan^{-1} x \Big|^{x = t}_{x = 1})\\
& = & \displaystyle \lim_{t \to \infty} (\tan^{-1} t - \tan^{-1} 1)\\
& = & \displaystyle \lim_{t \to \infty} \tan^{-1} t - \lim_{t \to \infty} \tan^{-1} 1\\
& = & \displaystyle \frac{\pi}{2} - \frac{\pi}{4}\\
& = & \displaystyle \frac{\pi}{4} \quad \text{converges}
\end{array}\]
\(\implies \displaystyle \sum^{\infty}_{n = 1} \frac{1}{n^2 +1}\) converges
\[\displaystyle \sum^{\infty}_{n = 1} \frac{1}{n^2 + 1} \neq \int^{\infty}_1 \frac{1}{x^2 + 1} dx\]
Compare with P-series\\
\[\displaystyle \sum^{\infty}_{n = 1} \frac{1}{n^2 + 1} < \sum^{\infty}_{n = 1} \frac{1}{n^2} = \frac{\pi^2}{6}\quad \text{converges}\]
\(\implies \displaystyle \sum^{\infty}_{n = 1} \) converges
\end{eg}
\begin{eg}
Determine \(\displaystyle \sum^{\infty}_{n =1} \frac{1}{n(ln n)^p}, p > 0\) converges or diverges

\soln
\[\displaystyle a_n = \frac{1}{n(ln n)^p}\]
\begin{itemize}
\item[(1)] \(a_n > 0 \quad \forall n = 2, 3, \cdots\)
\item[(2)] \(a_n \searrow\)
\item[(3)] \(\displaystyle f(x) = \frac{1}{x(ln x)^p}\) is conti for \(x \in [2, \infty)\)
\end{itemize}
\[\begin{array}{rcl}
\displaystyle \int^{\infty}_2 \frac{1}{x(ln x)^p} dx & = & \displaystyle \int^{x = \infty}_{x = 2} \frac{d lnx}{(lnx)^p}\\
& = & \displaystyle \int^{y = \infty}_{y = ln 2} \frac{dy}{y^p}\\
& = & \displaystyle \frac{1}{1 - p} y ^{1 - p} \Big|^{\infty}_{ln 2}\\
& = & \left\{\begin{array}{rcl}
\displaystyle \int^{\infty}_{ln2} \frac{1}{y} dy & = & lny \Big|^{\infty}_{ln2} = \infty, p = 1\\
\displaystyle \int^{\infty}_{ln2} \frac{1}{y^p} dy & = & \displaystyle \frac{1}{1 - p} y^{1 - p} \Big|^{\infty}_{ln2} = \left\{\begin{array}{ccl}
\text{converges} & , & p> 1\\
\text{diverges} & , & 0 < p < 1
\end{array} \right.\\
\end{array} \right.\\
& = & \left\{\begin{array}{ccl}
\text{diverges} & , & 0 < p \leq 1\\
\text{converges} & , & p >1
\end{array} \right.\\
\end{array}\]
\end{eg}
\begin{notn}
\[\displaystyle \int^{\infty}_1 \frac{1}{x^p} dx = \left\{ \begin{array}{ccl} \text{diverges} &, & 0 < p \leq 1\\
\text{converges} &, & p > 1 
\end{array}\right.\]
\[\displaystyle \sum^{\infty}_{n = 1} \frac{1}{n^p} = \left\{\begin{array}{ccl}
\text{diverges} &, & 0 < p \leq 1\\
\text{converges} &, & p > 1
\end{array} \right.\]
\[p = 1 \quad \text{Harmonic Series (調和級數)}\]
\end{notn}
\section{Comparison Theorem}
\begin{theorem}[Comparison Theorem]
\begin{itemize}
\item Subtraction 減法\\
\[a_n, b_n > 0 \quad a_n \leq b_n \quad n = 1, 2, \cdots\]
\begin{itemize}
\item[(1)] \(\displaystyle \sum^{\infty}_{n = 1} b_n \text{ converges } \implies \sum^{\infty}_{n = 1} a_n\) converges
\item[(2)] \(\displaystyle \sum^{\infty}_{n =1} a_n \text{ diverges } \implies \sum^{\infty}_{n = 1} b_n\) diverges 
\end{itemize}
\item Division 除法\\
\[\displaystyle \lim_{n \to \infty} \frac{a_n}{b_n} = c\]
\begin{itemize}
\item[(1)] \(c \neq 0 \text{ and } c \neq \infty\\
\displaystyle \sum^{\infty}_{n = 1} a_n \text{ and } \sum^{\infty}_{n = 1} b_n\) both converges or diverges
\item[(2)] \(c = 0 \quad (\displaystyle \lim_{n \to \infty} \frac{a_n}{b_n} = 0)\\
\displaystyle \sum^{\infty}_{n = 1} b_n \text{ converges } \implies \sum^{\infty}_{n = 1} a_n\) converges\\
\(\displaystyle \sum^{\infty}_{n = 1} a_n \text{ diverges } \implies \sum^{\infty}_{n = 1} b_n\) diverges 
\item[(3)] \(c = \infty\)\\
\(\displaystyle \sum^{\infty}_{n = 1} a_n \text{ converges } \implies \sum^{\infty}_{n = 1} b_n\) converges\\
\(\displaystyle \sum^{\infty}_{n = 1} b_n \text{ diverges } \implies \sum^{\infty}_{n = 1} a_n\) diverges \\
\end{itemize}
\end{itemize}
\end{theorem}
\begin{eg}
Determine \(\displaystyle \sum^{\infty}_{n = 1} \frac{1}{n^2 - n - 1}\) converges or diverges

\soln
\[\displaystyle a_n = \frac{1}{n^2 - n - 1} \quad b_n = \frac{1}{n^2} \quad (\sum^{\infty}_{n = 1} \frac{1}{n^2} \text{ converges)}\]
\[\displaystyle \lim_{n \to \infty} \frac{a_n}{b_n} = \lim_{n \to \infty} \frac{n^2}{n^2 - n - 1} = 1 \neq 0\]
\[\displaystyle \frac{1}{n^2 - n - 1} \geq \frac{1}{n^2} \quad n = 2, 3, \cdots\]
\(\implies \displaystyle \sum^{\infty}_{n = 1} a_n\) converges
\end{eg}
\begin{eg}
Determine \(\displaystyle \sum^{\infty}_{n = 1} \frac{1}{\sqrt{n(n + 1)(n + 2)}}\) converges or diverges

\soln
\[\displaystyle a_n = \frac{1}{\sqrt{n(n + 1)(n + 2)}} \quad b_n = \frac{1}{n^{\frac{3}{2}}} \quad (\sum^{\infty}_{n = 1} \frac{1}{n^{\frac{3}{2}}} \text{ converges)}\]
\[\displaystyle \lim_{n \to \infty} \frac{a_n}{b_n} = 1 \neq 0\]
\(\implies \displaystyle \sum^{\infty}_{n = 1} a_n\) converges
\end{eg}
\begin{eg}
Determine \(\displaystyle \sum^{\infty}_{n = 1} \sin (\frac{1}{n})\) converges or diverges

\soln
\[\displaystyle a_n = \sin (\frac{1}{n}) \quad b_n = \frac{1}{n} \quad (\sum^{\infty}_{n = 1} \frac{1}{n} \text{  diverges)}\]
\[\displaystyle \lim_{n \to \infty} \frac{a_n}{b_n} = \lim_{n \to \infty} \frac{\sin(\frac{1}{n})}{\frac{1}{n}} = \lim_{n \to \infty} \frac{\sin m}{m} = 1 \neq 0\]
\(\implies \displaystyle \sum^{\infty}_{n = 1} a_n\) diverges
\end{eg}
\begin{eg}
Determine \(\displaystyle \sum^{\infty}_{n = 1} \frac{n + 5}{\sqrt{n^7 + n^2}}\) converges or diverges

\soln
\[\displaystyle a_n = \frac{n + 5}{\sqrt{n^7 + n^2}} \quad b_n = \frac{1}{n^{\frac{4}{3}}} \quad (\sum^{\infty}_{n = 1} \frac{1}{n^{\frac{4}{3}}} \text{ converges)}\]
\[\displaystyle \lim_{n \to \infty} \frac{a_n}{b_n} = 1 \neq 0\]
\(\implies \displaystyle \sum^{\infty}_{n = 1} a_n\) converges
\end{eg}
\begin{eg}
Determine \(\displaystyle \sum^{\infty}_{n = 1} \frac{1}{n^{1 + \frac{1}{n}}}\) converges or diverges

\soln
\[\displaystyle a_n = \frac{1}{n^{1 + \frac{1}{n}}} \quad b_n = \frac{1}{n^1} \quad (\sum^{\infty}_{n = 1} \frac{1}{n} \text{ diverges)}\]
\[\displaystyle \lim_{n \to \infty} \frac{a_n}{b_n} = \lim_{n \to \infty} \frac{1}{n^{\frac{1}{n}}} = 1 \neq 0\]
\[\displaystyle \lim_{x \to \infty} x^{\frac{1}{x}} \quad (\text{type } \infty)\]
\[\displaystyle y = x^{\frac{1}{x}} \implies ln y = \frac{1}{x} lnx\]
\[\displaystyle \lim_{x \to \infty} \frac{lnx}{x} = \lim_{x \to \infty} \frac{\frac{1}{x}}{1} = 0\]
\[\displaystyle \lim_{x \to \infty} x^{\frac{1}{x}} = 1\]
\(\implies \displaystyle \sum^{\infty}_{n = 1} a_n\) diverges
\end{eg}
\begin{eg}
Determine \(\displaystyle \sum^{\infty}_{n = 1} \frac{n^3}{2^n}\) converges or diverges

\soln
\[\displaystyle a_n  = \frac{n^3}{2^n} = \frac{n^3}{(\frac{2}{1 \cdot 1})^n (1 \cdot 1)^n)} \quad b_n = \frac{1}{(\frac{2}{1 \cdot 1})^n} (\sum^{\infty}_{n = 1} \frac{1}{(\frac{2}{1 \cdot 1})^n} \text{ converges)}\] 
\[\text{geometric series 等比級數 with common ratio 公比} < 1\]
\[\displaystyle \lim_{n \to \infty} a_n = \lim_{n \to \infty} \frac{3n^2}{2^n ln2} = \frac{3}{ln2} \lim_{n \to \infty} \frac{n^2}{2^n} = \frac{3}{ln2} \lim_{n \to \infty} \frac{2n}{2^n ln2} = \frac{6}{(ln2)^2} \lim_{n \to \infty} \frac{1}{2^n} = 0\]
\[\displaystyle \lim_{n \to \infty} \frac{a_n}{b_n} = \lim_{n \to \infty} \frac{n^3}{(1 \cdot 1)^n} = 0\]
\(\implies \displaystyle \sum^{\infty}_{n = 1} a_n\) converges
\end{eg}
\section{Alternating Series 交錯級數}
\begin{defn}
\[\displaystyle \sum^{\infty}_{n = 1} (-1)^{n + 1} a_n = a_1 - a_2 + a_3 - a_4 + \cdots\]
Assume
\begin{itemize}
\item[(1)] \(a_n > 0 \quad \forall n = 1, 2, 3, \cdots\)
\item[(2)] \(a_n \searrow\)
\end{itemize}
\(\implies \displaystyle \sum^{\infty}_{n = 1} (-1)^{n + 1} a_n\) converges
\end{defn}
\begin{eg}
Determine \(\displaystyle \sum^{\infty}_{n = 1} (-1)^{n+1} \frac{1}{n} = \frac{1}{1} - \frac{1}{2} + \frac{1}{3} - \frac{1}{4} + \cdots\) converges or diverges

\soln
\begin{itemize}
\item[(1)] \(a_n = \frac{1}{n} > 0 \quad \forall n = 1, 2, \cdots\)
\item[(2)] \(a_n \searrow 0 \quad (a_n \searrow \text{ and } \displaystyle \lim_{n \to \infty} a_n = 0)\)
\end{itemize}
\(\implies \displaystyle \sum^{\infty}_{n = 1} (-1)^{n+1} a_n\) converges
\end{eg}
\begin{proof}
\[\begin{array}{rcl}
S_{2n} \nearrow \text{ has a upper bound } a_1 & \implies & \displaystyle \lim_{n \to \infty} S_{2n} \text{ converges}\\
S^*_{2n +1} \searrow \text{ has a lower bound } 0 & \implies & \displaystyle \lim_{n \to \infty} S^*_{2n + 1} \text{ converges}
\end{array}\]
Claim: \(\displaystyle \lim_{n \to \infty} S_{2n} = \lim_{n \to \infty} S^*_{2n + 1}\)\\
\[\displaystyle \lim_{n \to \infty} S_{2n} - \lim_{n \to \infty} S^*_{2n + 1} = \lim_{n \to \infty} (S_{2n} - S^*_{2n+ 1}) = 0\]
\[\displaystyle \lim_{n \to \infty} S_{2n} = \lim_{n \to \infty} S^*_{2n + 1}\]
\(\implies \displaystyle \sum^{\infty}_{n = 1} (-1)^{n +1} a_n\) converges
\end{proof}
\begin{eg}
Determine \(\displaystyle \sum^{\infty}_{n = 1} (-1)^n \cos (\frac{\pi}{n})\) converges or diverges

\soln
\begin{itemize}
\item[(1)] \(\displaystyle a_n = \cos (\frac{\pi}{n}) > 0 \quad \forall n = 3, 4, \cdots\)
\item[(2)] \(\displaystyle a_n \searrow \quad f(x) = \cos (\frac{\pi}{n}) \implies f'(x) = \)\\
\[\displaystyle \lim_{n \to \infty} \cos (\frac{\pi}{n}) = 1 \neq 0\]
\end{itemize}
\(\implies \displaystyle \sum^{\infty}_{n = 1} (-1)^n \cos (\frac{\pi}{n})\) diverges
\end{eg}
\begin{eg}
Determine \(\displaystyle \sum^{\infty}_{n = 1} (-1)^n \sin (\frac{\pi}{n})\) converges or diverges

\soln
\begin{itemize}
\item[(1)] \(\displaystyle a_n = \sin (\frac{pi}{n})>0 \quad \forall n = 2, 3, \cdots\)
\item[(2)] \(\displaystyle a_n \searrow \quad f(x) = \sin (\frac{\pi}{x}) \implies f'(x) = \cos (\frac{\pi}{x}) \cdot (\frac{-\pi}{x^2}) < 0, x \geq 3\)
\end{itemize}
\(\implies \displaystyle \sum^{\infty}_{n = 1} (-1)^n \sin (\frac{\pi}{n})\) converges
\end{eg}
\begin{eg}
Determine \(\displaystyle \sum^{\infty}_{n = 2} (-1)^n \frac{ln n}{n}\) converges or diverges?

\soln
\[\displaystyle a_n = \frac{ln n}{n}\]
\begin{itemize}
\item[(1)] \(a_n = \frac{ln n}{n} > 0 \quad \forall n = 2, 3, \cdots\)
\item[(2)] \(\displaystyle \lim_{n \to \infty} a_n = \lim_{n \to \infty} \frac{ln n}{n} = \lim_{n \to \infty} \frac{\frac{1}{n}}{1} = 0\)
\end{itemize}
let \(\displaystyle f(x) = \frac{lnx}{x}\)
\[\displaystyle f'(x) = \frac{x \frac{1}{x} - lnx}{x^2} = \frac{1}{x^2} (1 - lnx) \text{\quad \tssteelblue{Quotient Rule}}\]
\(\therefore \text{As } x > e \implies f'(x) < 0 \quad (ln e =1)\\
\implies a_n \nearrow \text{ when } n > 3\\
\implies \displaystyle \sum^{\infty}_{n = 2} (-1)^n \frac{ln n}{n}\) converges
\end{eg}
\begin{eg}
Determine \(\displaystyle \sum^{\infty}_{n = 2} (-1)^n \frac{(ln n)^p}{n}, p > 0\)  converges or diverges?

\soln
\[a_n = \frac{(ln n)^p}{n}\]
\begin{itemize}
\item[(1)] \(a_n = \frac{(ln n )^p}{n} > 0 \quad \forall n = 2, 3, \cdots\)
\item[(2)] \(\displaystyle \lim_{n \to \infty} a_n = \lim_{n \to \infty} \frac{(ln n)^p}{n}\)
\[\displaystyle \lim_{n \to \infty} \frac{p(ln n)^{p - 1} \frac{1}{n}}{1} = p (\lim_{n \to \infty} \frac{(ln n)^{p - 1})}{n})= \left\{\begin{array}{ccl}
0 & , & p \leq 1\\
\displaystyle \lim_{n \to \infty} \frac{(p - 1)(ln n)^{p - 2} \frac{1}{n}}{1} & , & p > 1
\end{array}\right.\]
\[\displaystyle p > 1: \quad p( p - 1) \lim_{n \to \infty} \frac{ln^{p - 2}}{n} = \left\{\begin{array}{ccl}
0 & , & 1 < p \leq 2\\
\cdots & , & p > 2
\end{array}\right.\]
\[\displaystyle \lim_{n \to \infty} \frac{(ln n)^p}{n} = 0\]
\end{itemize}
let \(\displaystyle f(x) = \frac{(ln x)^p}{x}\)
\[f'(x) =\frac{\cancel{x} p (ln x)^{p - 1} \cdot \frac{1}{\cancel{x}} - (ln x)^p \cdot 1}{x^2} = \frac{1}{x^2} (ln x)^{p - 1} (p - lnx)\]
\(\therefore \text{As } p - ln x < 0 \quad (x > e^p) \implies f'(x) < 0\)\\
\(\implies \displaystyle \sum^{\infty}_{n = 2} (-1)^n \frac{(ln n )^p}{n}\) converges
\end{eg}
\section{Absolute Convergence}
\begin{defn}[A.C and C.C]
\begin{itemize}
\item[(1)] If \(\displaystyle \sum^{\infty}_{n = 1} \Big| a_n \Big|\) converges, then \(\displaystyle \sum^{\infty}_{n = 1} a_n\) is absolute convergent (A.C).
\item[(2)] If \(\displaystyle \sum^{\infty}_{n = 1} \Big| a_n \Big|\) diverges, but \(\displaystyle \sum^{\infty}_{n = 1} a_n\) converges, then \(\displaystyle \sum^{\infty}_{n = 1} a_n\) is conditionally convergent (C.C).
\end{itemize}
\end{defn}
\begin{notn}
If \(\displaystyle \sum^{\infty}_{n = 1} a_n\) is A.C
\[\displaystyle \sum^{\infty}_{n = 1} \Big| a_n \Big| < \infty \text{\quad \tssteelblue{by def}}\] 
\[\displaystyle \Big| \sum^{\infty}_{n = 1} a_n \Big| \leq \sum^{\infty}_{n = 1} \Big| a_n \Big| < \infty\]
\[\displaystyle - \sum^{\infty}_{n = 1} \Big| a_n \Big| \leq \sum^{\infty}_{n = 1} a_n \leq \sum^{\infty}_{n = 1} \Big| a_n \Big|\]
\(\implies \displaystyle \sum^{\infty}_{n = 1} a_n\) converges
\end{notn}
\begin{eg}
Is \(\displaystyle \sum^{\infty}_{n = 1} (-1)^n \frac{1}{n^2}\) is A.C?

\soln
\[\displaystyle a_n = (-1)^n \frac{1}{n^2}\]
\(\displaystyle \sum^{\infty}_{n = 1} \Big| (-1)^n \frac{1}{n^2} \Big| = \sum^{\infty}_{n = 1} \frac{1}{n^2}\) is convergent \quad (\(\because p\)-series with \(\displaystyle p = 2)\)\\
\(\implies \displaystyle \sum^{\infty}_{n = 1} (-1)^n \frac{1}{n^2}\) is A.C)
\end{eg}
\begin{eg}
Is \(\displaystyle \sum^{\infty}_{n = 1} (-1)^n \frac{1}{n}\) is A.C?

\soln
\[\displaystyle a_n = (-1)^n \frac{1}{n}\]
\(\displaystyle \sum^{\infty}_{n = 1} \Big| (-1)^n \frac{1}{n} \Big| = \sum^{\infty}_{n = 1} \frac{1}{n}\) is divergent \quad (\(\because p\)-series with \(p = 1)\)\\\\\\
\(\displaystyle \sum^{\infty}_{n = 1} (-1)^n \frac{1}{n}\) is alternating series
\[\displaystyle a_n = \frac{1}{n}\]
\begin{itemize}
\item[(1)] \(\displaystyle a_n = \frac{1}{n} > 0 \quad \forall n = 1, 2, 3, \cdots\)
\item[(2)] \(a_n \searrow 0 \quad \displaystyle \lim_{n \to \infty} a_n = 0\)
\end{itemize}
\(\implies \displaystyle \sum^{\infty}_{n = 1} (-1)^n \frac{1}{n}\) converges\\\\\\
\(\implies \displaystyle \sum^{\infty}_{n = 1} (-1)^n \frac{1}{n}\) is C.C
\end{eg}
\section{Ratio and Root Tests}
\begin{defn}
Consider \(\displaystyle \sum^{\infty}_{n = 1} a_n\)
\[\left\{\begin{array}{cr}
\text{Ratio Test} & \displaystyle \lim_{n \to \infty} \Big| \frac{a_{n + 1}}{a_n} \Big| = L \geq 0\\
\text{Root Test} & \displaystyle \lim_{n \to \infty} \Big| a_n\Big| ^{\frac{1}{n}} = L \geq 0
\end{array}\right.\]
\begin{itemize}
\item[(1)] \(\displaystyle L < 1 \implies \sum^{\infty}_{n = 1} a_n\) is A.C (\(\displaystyle \implies\sum^{\infty}_{n = 1} \Big| a_n \Big|\) converges)
\item[(2)] \(\displaystyle L > 1 \implies \sum^{\infty}_{n = 1} a_n\) diverges (\(\displaystyle \implies \sum^{\infty}_{n = 1} \Big| a_n \Big|\) diverges)
\item[(3)] \(\displaystyle L = 1 \implies\) \underline{Inconclusive}\\
Any conclusion cannot be drawn from this test (i.e the test fails)
\end{itemize}
\end{defn}
\begin{eg}
Does \(\displaystyle \sum^{\infty}_{n = 1} \frac{n^2}{2^n}\) converge?

\soln
\[\displaystyle a_n = \frac{n^2}{2^n}\]
\begin{itemize}
\item Ratio Test\\
\[\displaystyle \lim_{n \to \infty} \Big| \frac{a_{n +1}}{a_n} \Big| = \lim_{n \to \infty} \frac{\frac{(n+1)^2}{2^{n +1}}}{\frac{n^2}{2^n}} = \lim_{n \to \infty} (\frac{(n + 1)^2}{n} \frac{1}{2} = \frac{1}{2} < 1\]
\(\implies \displaystyle \sum^{\infty}_{n = 1} \frac{n^2}{2^n}\) is A.C (\(\displaystyle \implies \sum^{\infty}_{n = 1} \frac{n^2}{2^n}\) converges)
\item Root Test\\
\[\displaystyle \lim_{n \to \infty} \Big| a_n \Big|^{\frac{1}{n}} = \lim_{n \to \infty} (\frac{n^2}{2^n})^{\frac{1}{n}} = \lim_{n \to \infty} \frac{n^{\frac{2}{n}}}{2} = \frac{1}{2} (\lim_{n \to \infty} n^{\frac{2}{n}}) = \frac{1}{2} < 1\]
\[\displaystyle f(x) = x^{\frac{2}{x}} \quad ln f(x) = \frac{2}{x} ln x\]
\[\displaystyle 2 \lim_{n \to \infty} \frac{ln x}{x} = 2 \lim_{n \to \infty} \frac{\frac{1}{x}}{1} = 0\]
\[\displaystyle \lim_{n \to \infty} x^{\frac{2}{x}}= 1\]
\(\displaystyle \implies \sum^{\infty}_{n = 1} \frac{n^2}{2^n}\) is A.C
\end{itemize}
\end{eg}
\begin{eg}
Does \(\displaystyle \sum^{\infty}_{n = 1} (\frac{n^2 + 1}{2n^2 + 1})^n\) converge?

\soln
\[\displaystyle a_n = (\frac{n^2 + 1}{2n^2 + 1})^n\]
\begin{itemize}
\item Root Test
\[\displaystyle \lim_{n \to \infty} ((\frac{n^2 + 1}{2n^2 + 1})^{\cancel{n}})^{\cancel{\frac{1}{n}}} = \lim_{n \to \infty} \frac{1n^2 + 1}{2n^2 + 1} = \frac{1}{2} < 1\]
\(\displaystyle \implies \sum^{\infty}_{n = 1} (\frac{n^2 + 1}{2n^2 + 1})\) is A.C
\end{itemize}
\end{eg}
\begin{eg} 
Does \(\displaystyle \sum^{\infty}_{n = 1} \frac{(-3)^n}{n!}\) converges?

\soln
\[\displaystyle a_n = \frac{(-3)^n}{n!}\]
\begin{itemize}
\item Ratio Test
\[\displaystyle \lim_{n \to \infty} \Big| \frac{a_{n +1}}{a_n} \Big| = \lim_{n \to \infty} \Bigg| \frac{\frac{(-3)^{n + 1}}{(n+1)!}}{\frac{(-3)^n}{n!}} \Bigg| = \lim_{n \to \infty} \frac{3}{n + 1} = 0 < 1\]
\(\displaystyle \implies \sum^{\infty}_{n \to \infty} \frac{(-3)^n}{n!}\) is A.C (\(\displaystyle \implies \sum^{\infty}_{n = 1} \frac{(-3)^n}{n!}\) converges)
\end{itemize}
\end{eg}
\begin{eg}
Is \(\displaystyle \sum^{\infty}_{n = 1} (-1)^n \frac{2^n n!}{5 \cdot 8 \cdot 11 \cdots (3n + 2)}\) A.C?

\soln
\[\displaystyle a_n = \frac{(-1)^n 2^n n!}{5 \cdot 8 \cdot 11 \cdots (3n + 2)}\]
\begin{itemize}
\item Ratio Test
\[\displaystyle \lim_{n \to \infty} \Big| \frac{a_{n + 1}}{a_n} \Big| = \lim_{n \to \infty} \Bigg|\frac{\frac{\cancel{(-1)^{n + 1}} 2^{n + 1} (n + 1)!}{\cancel{5 \cdot 8 \cdot 11 \cdots (3n +2)}(3n + 5)}}{\frac{\cancel{(-1)^n} 2^n n!}{\cancel{5 \cdot 8 \cdot 11 \cdots (3n + 2)}}} \Bigg| = \lim_{n \to \infty} \Big| \frac{2(n +1)}{3n + 5} \Big| = \frac{2}{3} < 1\]
\(\displaystyle \implies \sum^{\infty}_{n = 1} a_n\) is A.C \quad (i.e \(\displaystyle \sum^{\infty}_{n = 1} \Big| a_n \Big| < \infty\))
\end{itemize}
\end{eg}
\begin{eg}
Is \(\displaystyle \sum^{\infty}_{n = 1} \frac{(-1)^n 5^n n!}{5 \cdot 8 \cdot 11 \cdots (3n +2)}\) A.C?

\soln
\[\displaystyle a_n = \frac{(-1)^n 5^n n!}{5 \cdot 8 \cdot 11 \cdots (3n + 2)}\]
\begin{itemize}
\item Ratio Test
\[\displaystyle \lim_{n \to \infty} \Big| \frac{a_{n + 1}}{a_n} \Big| = \lim_{n \to \infty} \Bigg| \frac{\frac{\cancel{(-1)^{n +1}} 5^{n +1} (n + 1)!}{\cancel{5 \cdot 8 \cdot 11 \cdots (3n + 2)}(3n + 5)}}{\frac{\cancel{(-1)^n} 5^n n!}{\cancel{5 \cdot 8 \cdot 11 \cdots (3n + 2)}}} \Bigg| = \lim_{n \to \infty} \Big| \frac{5(n + 1)}{3n + 5} \Big| = \frac{5}{3} > 1\]
\(\displaystyle \implies \sum^{\infty}_{n = 1} a_n\) diverges (NOT A.C)
\end{itemize}
\end{eg}
\begin{eg}
Is \(\displaystyle \sum^{\infty}_{n = 1} \frac{(-1)^n}{(\tan^{-1} n)^n}\) A.C?

\soln
\[\displaystyle a_n = \frac{(-1)^n}{(\tan^{-1} n)^n}\]
\begin{itemize}
\item Root Test
\[\displaystyle \lim_{n \to \infty} \Big| a_n \Big|^{\frac{1}{n}} = \lim_{n \to \infty} \Big| \frac{1}{(\tan^{-1} n)^n} \Big|^{\frac{1}{n}} = \lim_{n \to \infty} \frac{1}{| \tan^{-1} n |} = \frac{1}{\frac{\pi}{2}} = \frac{2}{\pi} < 1\]
\(\displaystyle \implies \sum^{\infty}_{n = 1} a_n\) is A.C
\end{itemize}
\end{eg}
\section{Power Series 冪級數}
\begin{defn}
\[\displaystyle\sum^{\infty}_{n = 1} c_n (x - a)^n = c_0 (x - a)^0 + c_1 (x - a)^1 + c_2 (x - a)^2 + \cdots\]
\(\displaystyle\sum^{\infty}_{n = 1} c_n (x - a)^n\) is a power series about \(a\) (centered at \(a\)) or a power series in \((x - a)\)
\end{defn}
\begin{notn}
\begin{itemize}
\item ``Power'' means ``\(n\)''
\item Not a polynomial 
\item variable \(x\) in \(a\) series
\end{itemize}
\end{notn}
\begin{eg} Find the interval of convergence (收斂區間) and the ratio of convergence of \(\displaystyle \sum^{\infty}_{n = 1} \frac{(x - 3)^n}{n}\)

\soln
\[\displaystyle b_n = \frac{(x - 3)^n}{n} \quad \quad a = 3\]
\[c_n = \left\{\begin{array}{rcl}
\displaystyle \frac{1}{n} & , & n = 1, 2, 3 \cdots\\
0 & , & n = 0
\end{array}\right.\]
Use Ratio Test
\[\displaystyle \lim_{n \to \infty} \Big| \frac{b_{n + 1}}{b_n} \Big| = \lim_{n \to \infty} \Bigg| \frac{\frac{(x - 3)^{n + 1}}{n + 1}}{\frac{(x - 3)^n}{n}} \Bigg| = \lim_{n \to \infty} \Big| \frac{(x - 3)n}{n + 1} \Big| = |x - 3| \lim_{n \to \infty} \Big| \frac{n}{n + 1} \Big| = |x - 3| = L\]
\begin{itemize}
\item \(|x - 3| < 1 \implies \displaystyle \sum^{\infty}_{n = 1} b_n\) is A.C
\item \(|x - 3| > 1 \implies \displaystyle \sum^{\infty}_{n = 1} b_n\) is divergent
\item \(|x - 3| = 1 \implies\) Inconclusive
\begin{itemize}
\item[(1)] \(x = 2 \quad \displaystyle \sum^{\infty}_{n = 1} \frac{(-1)^n}{n} \left(\begin{array}{l}
\text{alternating series}\\
\displaystyle k_n = \frac{1}{n} \searrow 0
\end{array}\right)\)\\
\(\implies \displaystyle \sum^{\infty}_{n = 1} \frac{(-1)^n}{n}\) converges 
\item[(2)] \(x = 4 \quad \displaystyle \sum^{\infty}_{n = 1} \frac{1^n}{n} = \sum^{\infty}_{n = 1} \frac{1}{n}\) \quad (p-series with \(p = 1\))\\
\(\implies\) diverges 
\end{itemize}
\end{itemize}
I.O.C is \( 2 \leq x < 4\) or \([2, 4)\) and R.O.C \(=1\)
\end{eg}
\begin{eg}[Bessel Function]
Find the I.O.C and R.O.C of \(\displaystyle \sum^{\infty}_{n = 1} \frac{(-1)^nx^{2n}}{2^{2n}(n!)^2}\)

\soln
\[\displaystyle b_n = \frac{(-1)^n(x - 0)^{2n}}{2^{2n}(n!)^2}\]
Use Ratio Test
\[\displaystyle \lim_{n \to \infty} \Big| \frac{b_{n + 1}}{b_n} \Big| = \lim_{n \to \infty} \Bigg| \frac{\frac{\cancel{(-1)^{n + 1}}x^{2n + 2}}{2^{2n + 2}((n + 1)!)^2}}{\frac{\cancel{(-1)^n}x^{2n}}{x^{2n}(n!)^2}} \Bigg| = \lim_{n \to \infty} \Big| \frac{x^2}{4(n + 1)^2} \Big| = x^2 (\lim_{n \to \infty} \frac{1}{4(n + 1)^2}) = 0 \quad \forall x \in R\]
I.O.C is \((- \infty, \infty)\) and R.O.C = \(\infty\)
\end{eg}
\begin{eg}
Find the I.O.C of \(\displaystyle \sum^{\infty}_{n = 1} \frac{2^n(x - 1)^n}{n}\)

\soln
\[\displaystyle b_n = \frac{2^n(x - 1)^n}{n}\]
Use Ratio Test
\[\displaystyle \lim_{n \to \infty} \Big| \frac{b_{n + 1}}{b_n} \Big| = \lim_{n \to \infty} \Bigg| \frac{\frac{2^{n + 1} (x - 1)^{n +1}}{n + 1}}{\frac{2^n(x - 1)^n}{n}} \Bigg| = \lim_{n \to \infty} \Big| \frac{2(x - 1)n}{n + 1} \Big| = |x - 1| \lim_{n \to \infty} \frac{2n}{n + 1} = 2 |x - 1|\]
\begin{itemize}
\item \(2|x - 1| < 1 \implies \displaystyle \sum^{\infty}_{n = 1} b_n\) is A.C
\item \(2|x - 1| > 1 \implies \displaystyle \sum^{\infty}_{n = 1} b_n\) is divergent 
\item \(2|x - 1| = 1 \implies\) Inconclusive
\begin{itemize}
\item[(1)] \(\displaystyle x = \frac{1}{2} b_n = \frac{2^n(-\frac{1}{2})^n}{n} = \frac{(-1)^n}{n}\\
\implies \displaystyle \sum^{\infty}_{n = 1} b_n\) converges 
\item[(2)] \(\displaystyle x = \frac{3}{2} b_n = \frac{2^n(\frac{1}{2})^n}{n} = \frac{1}{n}\\
\implies \displaystyle \sum^{\infty}_{n = 1} b_n\) diverges
\end{itemize}
\end{itemize}
I.O.C is \(\displaystyle \frac{1}{2} \leq x < \frac{3}{2}\)
\end{eg}
\begin{eg}
Find the I.O.C and R.O.C of \(\displaystyle \sum^{\infty}_{n = 1} n!x^n\)
 
\soln
\[\displaystyle b_n =n!x^n\]
Use Ratio Test
\[\displaystyle \lim_{n \to \infty} \Big| \frac{b_{n + 1}}{b_n} \Big| = \lim_{n \to \infty} \Big| (n + 1)x \Big| = |x| \lim_{n \to \infty} (n + 1)\]
I.O.C is \(\{0\}\) and R.O.C \(= 0\)
\end{eg}
\section{Representation of Functions as Power Series}
Q: Does a function has a power series representation?
\[\displaystyle 1 + x + x^2 + \cdots = \sum^{\infty}_{n = 0} x^n = \frac{1}{1 - x} \quad \text{ as } |x| < 1\]
\begin{eg}
\[\displaystyle \frac{1}{1 + x^2} = \frac{1}{1 - (-x^2)} = \sum^{\infty}_{n = 0} (-x^2)^n \quad \text{ as } |-x^2| < 1\]
\begin{itemize}
\item \(\displaystyle \sum^{\infty}_{n = 0} (-x^2)^n = \sum^{\infty}_{n = 0} (-1)^n x^{2n}\)
\item \(\displaystyle |-x^2| < 1 \implies x^2 < 1 \implies -1 < x < 1 \text{ or } |x| < 1\)
\end{itemize}
\end{eg}
\begin{eg}
\[\displaystyle \frac{1}{2 - x} = \frac{1}{2(1 - \frac{x}{2})} = \frac{1}{2} \frac{1}{1 - \frac{x}{2}} = \frac{1}{2} \sum^{\infty}_{n = 0} (\frac{x}{2})^n \quad \text{ as } \Big| \frac{x}{2} \Big| < 1\]
\begin{itemize}
\item \(\displaystyle \frac{1}{2} \sum^{\infty}_{n = 0} (\frac{x}{2})^n = \sum^{\infty}_{n = 0} 2^{-n-1} x^n\)
\item \(\displaystyle \Big| \frac{x}{2} \Big| < 1 \implies |x| < 2 \implies -2 < x < 2\)
\end{itemize}
\end{eg}
\begin{eg}
\[\begin{array}{rcl}
\displaystyle \frac{x^3}{2 + x^2} & = & \displaystyle x^3 (\frac{1}{x + x^2})\\
& = & \displaystyle x^3 \frac{1}{2(1 + \frac{x^2}{2})}\\
& = & \displaystyle \frac{x^3}{2} \frac{1}{1 - (- \frac{x^2}{2})}\\
& = & \displaystyle \frac{x^3}{2} \sum^{\infty}_{n = 0} (- \frac{x^2}{2})^n\\
& = & \displaystyle \frac{x^3}{2} \sum^{\infty}_{n = 0} (-1)^n \frac{1}{2^n} x^{2n}\\
& = & \displaystyle \sum^{\infty}_{n = 0} \frac{(-1)^n}{2^{n + 1}} x^{2n + 3}
\end{array}\]
\end{eg}
\section{Term by Term Differentiation and Integration}
\begin{theorem}
If \(\displaystyle \sum^{\infty}_{n = 0} c_n (x - a)^n\) has I.O.C \(|x - a| < r\) and let \(\displaystyle f(x) = \sum^{\infty}_{n = 0} c_n (x - a)^n\), then
\begin{itemize}
\item[(1)] Differentiation
\[\begin{array}{rcl}
\displaystyle f'(x) & = & \displaystyle  \frac{d}{dx} (\sum^{\infty}_{n = 0} c_n (x - a)^n\\
& = & \displaystyle \sum^{\infty}_{n = 0} \frac{d}{dx}(c_n (x - a)^n) \quad x \in (a - r, a + r)\\
& = & \displaystyle \sum^{\infty}_{n = 0} c_n \frac{d}{dx} ((x- a)^n)\\
& = & \displaystyle 0 + c_1 + 2c_2(x - a)^1 + \cdots\\
& = & \displaystyle \sum^{\infty}_{n = 1} c_n n(x - a)^{n - 1}
\end{array}\]
\item[(2)] Integration
\[\begin{array}{rcl}
\displaystyle \int f(x) dx & = & \displaystyle \int \sum^{\infty}_{n = 0} c_n (x - a)^n dx\\
& = & \displaystyle \sum^{\infty}_{n = 0} (\int c_n (x - a)^n dx)\\
& = & \displaystyle \sum^{\infty}_{n =0} (c_n \frac{1}{n + 1} (x - a)^{n + 1}) + C
\end{array}\]
\end{itemize}
\end{theorem}
\begin{eg}
Find p.s.r of \(\displaystyle f(x) = ln (1 - x)\)

\soln
\[\displaystyle f'(x) = \frac{-1}{1 - x} = \sum^{\infty}_{n = 0} - x^n \quad \text{ as } |x| < 1\]
\[ \displaystyle \int f'(x) dx = \int (\sum^{\infty}_{n = 0} -x^n) dx = - \sum^{\infty}_{n = 0} (\int x^n dx) = - \sum^{\infty}_{n = 0} \frac{1}{n + 1} x^{n + 1} + C \quad \tssteelblue{|x| < 1}\]
\[\displaystyle f(x) = - \sum^{\infty}_{n = 0} \frac{1}{n + 1} x^{n + 1} + C \quad \tssteelblue{|x| < 1}\]
let \(\displaystyle x = 0 \implies ln 1 0 + C \implies C = 0\)
\[ln (1 - x) = \sum^{\infty}_{n = 0} \frac{-1}{n + 1} x^{n + 1} \quad \tssteelblue{|x| < 1}\]
\end{eg}
\begin{eg}
Find p.s.r of \(\displaystyle f(x) = ln(1 - x)\)

\soln
\[\begin{array}{lcll}
\displaystyle f'(x) = \frac{-1}{1 - x} & = & \displaystyle - \sum^{\infty}_{n = 0} x^n & \quad |x|< 1\\
\displaystyle \int f'(x)dx & = & \displaystyle \int - \sum^{\infty}_{n = 0} x^n dx\\
& = & \displaystyle - \sum^{\infty}_{n = 0} (\int x^n dx) & \quad |x| < 1\\
& = & \displaystyle - \sum^{\infty}_{n = 0} \frac{1}{n + 1}x^{n + 1} & \quad |x| < 1\\
\displaystyle ln(1 - x) + C & = & \displaystyle - \sum^{\infty}_{n = 0} \frac{1}{n + 1} x^{n + 1} & \quad |x| < 1
\end{array}\]
let \( x = 0\)
\[\displaystyle ln 1 + C = 0 \implies C = 0\]
\[\displaystyle ln (1 - x) = \sum^{\infty}_{n = 0} \frac{-1}{n + 1} x^{n + 1}\]
\end{eg}
\begin{eg}
Find p.s.r of \(f(x) = \tan^{-1} x\)

\soln
\[\begin{array}{rcll}
\displaystyle f'(x) & = &\displaystyle \frac{1}{1 + x^2}\\
& = & \displaystyle \frac{1}{1 - 1(-x^2)}\\
& = & \displaystyle \sum^{\infty}_{n = 0} (-x^2) & \quad |-x^2| < 1\\
& = & \displaystyle \sum^{\infty}_{n = 0} ( -1)^n x^{2n} & \quad |x|< 1
\end{array}\]
\[\begin{array}{lcl}
\displaystyle \int f'(x) dx & = & \displaystyle \int \sum^{\infty}_{n = 0} (-1)^n x^{2n} dx\\
& = & \displaystyle \sum^{\infty}_{n = 0} (\int (-1)^n x^{2n} dx)\\
& = & \displaystyle \sum^{\infty}_{n = 0} (-1)^n \frac{1}{2n + 1} x^{2n + 1} + C\\
\displaystyle \tan^{-1} x & = & \displaystyle \sum^{\infty}_{n = 0} \frac{(-1)^n}{2n + 1} x^{2n + 1} + C
\end{array}\]
let \(x = 0\)
\[\displaystyle \tan^{-1} 0 = 0 + C \implies C = 0\]
\[\displaystyle \tan^{-1} x = \sum^{\infty}_{n = 0} \frac{(-1)^n}{2n + 1} x^{2n + 1} = x - \frac{x^3}{3} + \frac{x^5}{5} - \frac{x^7}{7} + \cdots\]
let \(x = 1\)
\[\displaystyle \tan^{-1} 1 = 1 - \frac{1}{3}  + \frac{1}{5} - \frac{1}{7} + \cdots + \frac{d}{dx} (1 - x)^{-1} = ( 1 - x)^{-2}\]
\end{eg}
\begin{eg}
Find p.s.r of \(\displaystyle f(x) = \frac{1}{(1 - x)^2}\)

\soln
\[\begin{array}{lcll}
\displaystyle \int f(x) dx & = & \displaystyle \int ( 1-x)^{-2} dx\\
& = & \displaystyle \frac{1}{1 - x} + C\\
& = & \displaystyle \sum^{\infty}_{n = 0} x^n + C & \quad |x| < 1
\end{array}\]
\[\begin{array}{lcll}
\displaystyle \frac{d}{dx} \int f(x) dx & = & \displaystyle \frac{d}{dx} \sum^{\infty}_{n = 0} x^n + C & \quad |x| < 1\\
& = & \displaystyle \frac{d}{dx} (\sum^{\infty}_{n = 0} x^n) + \frac{d}{dx} C\\
& = & \displaystyle \sum^{\infty}_{n = 0} (\frac{d}{dx} x^n)\\
& = & \displaystyle \sum^{\infty}_{n = 1} nx^{n - 1}\\
\displaystyle \frac{1}{(1 - x)^2} & = & \displaystyle \sum^{\infty}_{n = 1} nx^{n - 1} & \quad |x| < 1
\end{array}\]
\end{eg}
\begin{eg}
Does \(\displaystyle \sum^{\infty}_{n = 1} (-1)^n (\sqrt{x + 1} - \sqrt{n})\) converge?

\soln
\[\displaystyle a_n = \sqrt{n + 1} - \sqrt{n} < 0 \quad n = 1, 2, \cdots\]
\begin{itemize}
\item \(\displaystyle \lim_{n \to \infty} a_n = \lim_{n \to \infty} (\sqrt{n + 1} - \sqrt{n}) = \lim_{n \to \infty} \frac{(\sqrt{n + 1} - \sqrt{n})(\sqrt{n + 1} + \sqrt{n})}{(\sqrt{n + 1} + \sqrt{n})} = 0\)
\item \(\displaystyle a_n = \frac{1}{\sqrt{n + 1} + \sqrt{n}} \searrow\)
\end{itemize}
\(\implies \displaystyle \sum^{\infty}_{n = 1} ( -1)^n a_n\) converges
\end{eg}
\begin{eg}
Find p.s.r of \(\displaystyle \int \frac{dx}{1 + x^7}\)

\soln
\[\begin{array}{lcll}
\displaystyle f(x) & = & \displaystyle \int \frac{dx}{1 + x^7}\\
\displaystyle f'(x) & = & \displaystyle \frac{d}{dx} \int \frac{dx}{1 + x}\\
& = & \displaystyle \frac{1}{1 + x^7}\\
& = & \displaystyle \frac{1}{1 - (-x^7)}\\
& = & \displaystyle \sum^{\infty}_{n = 0} (-x^7)^n & \quad |-x^7| < 1\\
& = & \displaystyle \sum^{\infty}_{n = 0} (-1)^n x^{7n} & \quad |x| < 1
\end{array}\]
\[\begin{array}{lcll}
\displaystyle \int f'(x)dx & = & \displaystyle \int \sum^{\infty}_{n = 0} (-1)^n x^{7n} & \quad |x| < 1\\
f(x) & = & \displaystyle \sum^{\infty}_{n = 0} (-1)^n \int x^{7n} dx + C & \quad |x| < 1\\
& = & \displaystyle \sum^{\infty}_{n = 0} \frac{(-1)^n}{7n + 1} x^{7n + 1} + C & \quad |x| < 1
\end{array}\]
\end{eg}
\section{Taylor and Maclaurin Series}
\begin{theorem}
Assume \(\displaystyle f(x) = \sum^{\infty}_{n = 0} c_n (x - a)^n\) has p.s.r, then 
\[\displaystyle c_n = \frac{f^{(n)}(a)}{n!}\]
\end{theorem}
\[\displaystyle f(x) = c_0 + c_1 (x - a) + c_2 (x - a)^2 + c_3 (x - a)^3 + c_4 (x - a)^4 + \cdots\]
\begin{itemize}
\item[(1)] let \(x = a \implies \displaystyle c_0 = f(a) = \frac{f(a)}{0!}\)
\item[(2)] \(\displaystyle f'(x) = 0 + c_1 + 2c_2(x - a) + 3c_3(x - a)^2 + 4c_4(x - a)^3 + \cdots\)\\
let \(x = a \implies \displaystyle c_1 = f'(a) = \frac{f'(a)}{1!}\)
\item[(3)] \(\displaystyle f''(x) = 0 + 2c_2 + 3 \cdot 2 c_3(x - a) + 4 \cdot 3 c_4(x - a) + \cdots\)\\
let \(x = a \implies \displaystyle c_2 = \frac{1}{2} f''(a) = \frac{f''(a)}{2!}\)
\item[(4)] \(\displaystyle f'''(x) = 0 + 3 \cdot 2 c_3 + 4 \cdot 3 \cdot 2 c_4(x - a) + \cdots\)\\
let \(x = a \implies \displaystyle c_3 = \frac{f'''(a)}{3!}\)\\
\vdots
\item[] \(\displaystyle c_n = \frac{f^{(n)}(a)}{n!}\)
\end{itemize}
\begin{defn}
\begin{itemize}
\item[(1)] \(\displaystyle \sum^{\infty}_{n = 0} c_n (x - a)^n \text{ , whose } c_n \frac{f^{(n)}(a)}{n!}\), is called Taylor series of \(f(x)\) about \(a\)\\
(Given \(f(x)\) and \(a \implies\) Taylor series of \(f(x)\) of \(a\))
\item[(2)] \(\displaystyle \sum^{\infty}_{n = 0} c_nx^n\) with \(\displaystyle c_n = \frac{f^{(n)}(0)}{n!}\) is called Maclaurin series of \(f(x)\)\\
\end{itemize}
\end{defn}
\begin{eg}
Find Maclaurin series of \(\displaystyle f(x) = e^x\)

\soln
\[\begin{array}{rcl}
f(x) & = & e^x\\
f'(x) & = & e^x\\
& \vdots &\\
f^{(n)}(x) & = & e^x\\
f^{(n)}(0) & = & 1
\end{array}\]
\[\displaystyle c_n  = \frac{f^{(n)}(0)}{n!} = \frac{1}{n!}\]
\(\therefore\) Maclaurin series of \(f(x) = e^x\) is 
\[\displaystyle \sum^{\infty}_{n = 0} \frac{x^n}{n!} = \frac{x^0}{0!} + \frac{x^1}{1!} + \frac{x^2}{2!} + \frac{x^3}{3!} + \cdots + \frac{x^n}{n!}\]
\[e^x = 1 + x + \frac{x^2}{2} + \frac{x^3}{6} + \cdots\]
\end{eg}
\begin{eg}
Find Maclaurin series of \(f(x) = \sin x\)

\soln
\[\begin{array}{rcrcrcr}
f(x) & = & \sin x & \implies & f(0) & = & 0\\
f'(x) & = & \cos x & \implies & f'(0) & = & 1\\
f''(x) & = & -\sin x & \implies & f''(0) & = & 0\\
f'''(x) & = & -\cos x & \implies & f'''(0) & = & -1\\
f^{(4)}(x) & = & \sin x & \implies & f^{(4)}(0) & = & 0
\end{array}\]
\[\displaystyle c_n = \frac{(-1)^n}{(2n + 1)!}\]
\(\therefore\) Maclaurin series of \(f(x) = \sin x\) is 
\[\displaystyle \sum^{\infty}_{n = 0} \frac{(-1)^n}{(2n + 1)!}x^{2n + 1} = \frac{1}{1!} x - \frac{1}{3!} x^3 + \frac{1}{5!} x^5 - \frac{1}{7!} x^7 + \cdots (= \sin x)\]
\[\displaystyle \sin x = x - \frac{x^3}{3!} + \frac{x^5}{5!} + \cdots = \sum^{\infty}_{n = 0} \frac{(-1)^n}{(2n + 1)!} x^{2n + 1}\]
\begin{itemize}
\item[(1)] \(\sin (-x) = - \sin x\)
\item[(2)] \(\displaystyle \lim_{n \to \infty} \frac{\sin x}{x} = 1\)
\end{itemize}
\end{eg}
\begin{eg}
Find Maclaurin series of \(f(x) = \cos x\)

\soln
\[\begin{array}{rcrcrcr}
f(x) & = & \cos x & \implies & f(0) & = & 1\\
f'(x) & = & -\sin x & \implies & f'(0) & = & 0\\
f''(x) & = & -\cos x & \implies & f''(0) & = & -1\\
f'''(x) & = & \sin x & \implies & f'''(0) & = & 0\\
f^{(4)}(x) & = & \cos x & \implies & f^{(4)}(0) & = & 1
\end{array}\]
\[\begin{array}{rcl}
\displaystyle \sum^{\infty}_{n = 0} \frac{f^{(n)}(0)}{n!} & = & \displaystyle \frac{f(0)}{0!} + \frac{f'(0)}{1!}x + \frac{f''(0)}{2!}x^2 + \frac{f'''(0)}{3!}x^3 + \cdots\\
& = & \displaystyle 1 - \frac{x^2}{2!} + \frac{x^4}{4!} - \frac{x^6}{6!} + \cdots + \frac{(-1)^n}{(2n)!} x^{2n} + \cdots\\
& = & \displaystyle \sum^{\infty}_{n = 0} \frac{(-1)^n}{(2n)!} x^{2n}\\
& = & \cos x
\end{array}\]
Q: What is the I.O.C of \(\displaystyle \sum^{\infty}_{n = 0} \frac{(-1)^n}{(2n + 1)!} x^{2n + 1}\)?
\[\displaystyle a_n = \frac{(-1)^n}{(2n + 1)!} x^{2n + 1}\]
Use Ratio Test
\[\displaystyle \lim_{n \to \infty} \Big| \frac{a_{n + 1}}{a_n} \Big| = \lim_{n \to \infty} \Bigg| \frac{\frac{\cancel{(-1)^{n + 1}}}{(2n + 3)!}x^{2n + 3}}{\frac{\cancel{(-1)^n}}{(2n + 1)!}x^{2n + 1}}\Bigg| = \lim_{n \to \infty} \Big| \frac{x^2}{(2n + 2)(2n + 3)} \Big| = x^2 (\lim_{n \to \infty} \frac{1}{(2n + 2)(2n + 3)}) = 0\]
\end{eg}
\begin{lemma}
\[\displaystyle \lim_{n \to \infty} (n!)^{\frac{1}{n}} = \infty\]
\end{lemma}
\begin{eg}
Is \(\displaystyle \sum^{\infty}_{n = 0} \frac{(-3)^n}{n!}\) A.C?

\soln
Use Root Test
\begin{itemize}
\item \(\displaystyle a_n = \frac{(-3)^n}{n!}\)
\item \(\displaystyle \lim_{n \to \infty} |a_n|^{\frac{1}{n}} = \lim_{n \to \infty} \frac{3}{(n!)^{\frac{1}{n}}} = 0 < 1\)
\end{itemize}
\(\displaystyle \implies \sum^{\infty}_{n = 0}\) is A.C
\end{eg}
\begin{eg}
Find I.O.C of \(\displaystyle \sum^{\infty}_{n = 0} \frac{(-1)^n x^{2n}}{x^{2n}(n!)^2}\)

\soln
Use Root Test
\begin{itemize}
\item \(\displaystyle a_n = \frac{(-1)^n x^{2n}}{x^{2n}(n!)^2}\)
\item \(\displaystyle \lim_{n \to \infty} |a_n|^{\frac{1}{n}} = \lim_{n \to \infty} \frac{x^2}{4((n!)^{\frac{1}{n}})^2} = \frac{x^2}{4} \lim_{n \to \infty} \frac{1}{((n!)^{\frac{1}{n}})^2} = 0 \quad \forall x \in R < 1\)
\end{itemize}
\(\implies\) I.O.C is \(R\) or \((-\infty, \infty)\)
\end{eg}
\begin{eg}
Find I.O.C of \(\displaystyle \sum^{\infty}_{n = 0} n!x^n\)

\soln
Use Root Test
\begin{itemize}
\item \(\displaystyle a_n = n!x^n\)
\item \(\displaystyle \lim_{n \to \infty} |x|(n!)^{\frac{1}{n}} = (\lim_{n \to \infty} (n!)^{\frac{1}{n}}) |x| = \left\{ \begin{array}{rcl}
0 < 1 &, & x = 0\\
\infty > 1 &, & x \neq 0
\end{array} \right.\)
\end{itemize}
\(\implies\) I.O.C is \({0}\)
\end{eg}
\begin{eg}
Find M.S of \(\displaystyle f(x) = \left\{ \begin{array}{rcl}
\displaystyle e^{-\frac{1}{x^2}} &, & x \neq 0\\
0 &, & x = 0
\end{array} \right.\)

\soln
\[\displaystyle \sum^{\infty}_{n = 0} \frac{f^{(n)}(0)}{n!}x^n\]
\[\begin{array}{lcl}
f(0) & = & 0\\
f'(0) & = & \displaystyle \lim_{h \to \infty} \frac{f(0 + h) - f(0)}{h}\\
& = & \displaystyle \lim_{h \to \infty} \frac{e^{-\frac{1}{h^2}}}{h}\\
& = & \displaystyle \lim_{m \to \infty} \frac{e^{-m^2}}{\frac{1}{m}}\\
& = & \displaystyle \lim_{m \to \infty} \frac{m}{e^{m^2}}\\
& \overset{L}{=} & \displaystyle \lim_{m \to \infty} \frac{1}{2me^{m^2}}\\
& = & 0\\
f''(0) & = & \displaystyle \lim_{h \to \infty} \frac{f'(h) - f'(0)}{h}\\
& = & \displaystyle \lim_{h \to \infty} \frac{e^{- \frac{1}{h^2}}}{2h^4}\\
& = & 0\\
& \vdots &\\
f^{(n)}(0) & = & 0 \quad n = 1, 2, \cdots
\end{array}\]
M.S of \(f(x)\) is \(\displaystyle \sum^{\infty}_{n = 0} \frac{0}{n!}x^n = 0\) but \(f(x) \neq 0\)
\end{eg}
\begin{notn}
\[\begin{tabular}{c|cc}
& \text{限制} & \text{equal to Taylor series}\\\hline
11.9 & \text{Yes} & \text{一定}\\
11.10 & \text{No} & \text{不一定}
\end{tabular}\]
\end{notn}
\begin{eg}
\[\begin{array}{rcl}
\displaystyle \lim_{x \to 0} \frac{e^x - 1 - x}{x^2} & = & \displaystyle \lim_{x \to 0} \frac{(1 + x + \frac{x^2}{2} + \frac{x^3}{6} + \cdots) - 1 -x}{x^2} = \frac{1}{2}
\end{array}\]
\end{eg}
\begin{eg}
Find the first non-zero terms in the Maclaurin series of \(f(x)\)

\soln
\begin{itemize}
\item[(a)]\(f(x) = \tan x\)\\
\[\begin{array}{rcl}
\tan x & = & \displaystyle \frac{\sin x}{\cos x}\\
\sin x & = & \displaystyle x - \frac{x^3}{6} + \frac{x^5}{120} - \cdots \quad x \in R\\
\cos x & = & \displaystyle 1 - \frac{x^2}{2} + \frac{x^4}{24} - \cdots \quad x \in R\\
\end{array}\]
\[\begin{array}{rccccccccc}
&& x & + & \displaystyle \frac{x^3}{3} & + & \displaystyle \frac{2}{15} x^5\\ \cline{3-9}
\displaystyle 1 - \frac{x^2}{2} + \frac{x^4}{24} - \cdots & ) & x & - & \displaystyle \frac{x^3}{6} & + & \displaystyle \frac{x^5}{120} & - & \cdots\\
&& x & - & \displaystyle \frac{x^3}{2} & + & \displaystyle \frac{x^5}{24}\\ \cline{3-9}
&&&& \displaystyle \frac{x^3}{3} & - & \displaystyle \frac{x^5}{30}\\
&&&& \displaystyle \frac{x^3}{3} & - & \displaystyle \frac{x^5}{6} & + & \displaystyle \frac{x^7}{72}\\\cline{3-9}
&&&&&& \displaystyle \frac{x}{15} x^5
\end{array}\]
\item[(b)] \(f(x) = e^x \sin x\)\\
\[\begin{array}{rcl}
e^x & = & \displaystyle 1 + x + \frac{x^2}{2} + \frac{x^3}{6} + \cdots \quad x \in R\\
\sin x & = & \displaystyle x - \frac{x^3}{6} + \frac{x^5}{120} - \cdots \quad x \in R\\
e^x \sin x & = & \displaystyle (1 + x + \frac{x^2}{2} + \frac{x^3}{6} + \cdots) \cdot (x - \frac{x^3}{6} + \frac{x^5}{120} - \cdots)\\
& = & \displaystyle x + x^2 + x^3 (- \frac{1}{6} + \frac{1}{2})\\
& = & \displaystyle x + x^2 + \frac{x^3}{3} + \cdots \quad x \in R
\end{array}\]
\end{itemize}
\end{eg}
\begin{eg}
Find the Taylor series for \(f(x) = e^x\) at \(a = 2\)

\soln
\[\displaystyle \sum^{\infty}_{n = 0} \frac{f^{(n)}(2)}{n!}(x - 2)^{n} = \sum^{\infty}_{n = 0} \frac{e^2}{n!}(x - 2)^n = e^2 (\sum^{\infty}_{n = 0} \frac{1}{n!} (x - 2)^n)\]
Find I.O.C
\[\displaystyle \lim_{n \to \infty} (n!)^{\frac{1}{n}} = \infty\]
Use Root Test
\[\begin{array}{rcl}
\displaystyle \lim_{n \to \infty} \frac{|x - 2|}{(n!)^{\frac{1}{n}}} & = & \displaystyle |x - 2| \lim_{n \to \infty} \frac{1}{(n!)^{\frac{1}{n}}}\\
& = & \displaystyle |x - 2| \cdot 0\\
& = & 0 \quad \forall x \in R
\end{array}\]
\(\implies \text{I.O.C is } R \text{ or } (-\infty, \infty)\)
\end{eg}
\begin{eg}
Use the Maclaurin series for \(f(x) = \sin x\) to find the Maclaurin series for \(g(x) = \cos x\)

\soln
\[\begin{array}{rcl}
\sin x & = & \displaystyle x - \frac{x^3}{3!} + \frac{x^5}{5!} - \cdots \quad x \in R\\
& = & \displaystyle \sum^{\infty}_{n = 0} \frac{(-1)^n}{(2n + 1)!} x^{2n + 1} \quad x \in R\\
\displaystyle \int \sin x & = & \displaystyle \frac{d}{dx} (\sum^{\infty}_{n = 0} \frac{(-1)^n}{(2n + 1)!} x^{2n + 1})\\
& = & \displaystyle \sum^{\infty}_{n = 0} (\frac{(-1)^n}{(2n + 1)!} (2n + 1)x^2n)\\
& = & \displaystyle \sum^{\infty}_{n = 0} \frac{(-1)^n}{(2n)!} \quad x \in R
\end{array}\]
\end{eg}
\begin{defn}
\(\displaystyle Tn(x) = \sum^n_{k = 0} \frac{f^{(k)}(a)}{k!} (x - a)^k\): the n-th degree of Taylor polynomial of \(f\) at \(a\) 
\end{defn}
\section{Taylor Inequality}
\begin{defn}
If 
\[|f^{(n + 1)} (x)| \leq M \text{ for } |x - a| \leq d\] 
Then
\[\displaystyle |f(x) - Tn(x)| \leq \frac{M}{(n + 1)!} (x- a)^{n + 1} \text{ for } |x - a| \leq d\]
Where 
\[\displaystyle Tn(x) = \sum^{\infty}_{k = 0} \frac{f^{(k)} (a)}{k!} (x - a)^k\]
\(Rn(x) = f(x) - Tn(x)\): Remainder to of Taylor series of \(f\) at \(a\)
\end{defn}
\begin{eg}
Show that \(\displaystyle e^x = \sum^{\infty}_{n = 0} \frac{x^n}{n!}, x \in R\)

\soln
\begin{itemize}
\item I.O.C is \(R\)\\
Use Ratio Test
\[\displaystyle a_n = \frac{x^n}{n!}\]
\[\displaystyle \lim_{n \to \infty} \Big| \frac{a_{n + 1}}{a_n} \Big| = \lim_{n \to \infty} \Bigg| \frac{\frac{x^{n + 1}}{(n + 1)!}}{\frac{x^n}{n!}} \Bigg| = |x| \lim_{n \to \infty} \frac{1}{n + 1} = 0 < 1 \quad \forall x \in R\]
\item \(f(x) = e^x \implies f^{(n)} (x) = e^x \quad \forall x = 0, 1, 2, \cdots\)\\
When \(|x| \leq d\) 
\[ |e^x| \leq e^d = M\]
\[M = e^d, a = 0\] 
By Taylor Inequality,
\[\displaystyle \Big| f(x) - \sum^{\infty}_{k = 0} \frac{x^k}{k!} \Big| \leq \frac{e^d}{(n + 1)!} x^{n + 1}\]
Use the fact \(\displaystyle \sum^{\infty}_{n = 0} \frac{x^n}{n!}\) conveys \(\forall x \in R\)
\[\displaystyle \lim_{n \to \infty} a_n = 0 \quad \forall x \in R\]
\[\displaystyle \lim_{n \to \infty} \frac{x^n}{n!} = 0\]
\[\displaystyle \lim_{n \to \infty} \Big| f(x) - \sum^{\infty}_{k =0} \frac{x^k}{k!} \Big| = 0\]
\[\displaystyle \sum^{\infty}_{k = 0} \frac{x^k}{k!} = f(x) = e^x\]
\end{itemize}
\end{eg}
\begin{eg}
Show that \(\displaystyle (\infty)^{\frac{1}{\infty}} = \infty^0\)

\soln
\[\displaystyle \lim_{n \to \infty} (n!)^{\frac{1}{n}} = \infty\]
\[\displaystyle e^x = 1 + x + \frac{x^2}{2!} + \cdots + \frac{x^n}{n!} + \cdots \quad x \in R\]
let \(x = n\)
\[\begin{array}{rcl}
e^x & \geq & \displaystyle \frac{x^n}{n!} \quad x > 0\\
e^n & \geq & \displaystyle \frac{n^n}{n!}\\
(e^n)^{\frac{1}{n}} & \geq & \displaystyle \frac{(n^n)^{\frac{1}{n}}}{(n!)^{\frac{1}{n}}}\\
e & \geq & \displaystyle \frac{n}{(n^n)^{\frac{1}{n}}}\\
e(n!)^{\frac{1}{n}} & \geq & \displaystyle n\\
\displaystyle e \lim_{n \to \infty} (n!)^{\frac{1}{n}} & \geq & \displaystyle \lim_{n \to \infty} n\\
\displaystyle \lim_{n \to \infty} (n!)^{\frac{1}{n}} & \geq & \displaystyle \frac{\displaystyle \lim_{n \to \infty} n}{e} = \infty
\end{array}\]
\end{eg}
\subsection*{Show Taylor Inequality}
Idea F.T.C
\[\begin{array}{rcl}
f(x) & = & \displaystyle f(a) + \int^x_a f'(t)dt\\
& = & \displaystyle f(a) + \int^x_a f'(t)d(t-x)\\
& \overset{I.B.P}{=} & \displaystyle f(a) + f'(t)(t-x) \Big|^{t = x}_{t = a} - \int^x_a (t - x)f''(t)dt\\
& = & \displaystyle f(a) - f'(a)(a -x) - \int^x_a \frac{1}{2} f''(t) d((t-x)^2)\\
& \overset{I.B.P}{=} & \displaystyle f(a) - f'(a)(a - x) - \frac{1}{2} ( f''(t)(t - x)^2 \Big|^{t = x}_{t = a} - \int^x_a (t - x)^2 f'''(t)dt)\\
& = & \displaystyle f(a) + f'(a)(x - a) + \frac{1}{2}(1 - f''(a)(a - x)^2) + \frac{1}{2}\int^x_a (t - x)^2f'''(t)dt\\
& = & \displaystyle f(a) + f'(a)(x - a) + \frac{1}{2} f''(a)(x - a)^2 + \frac{1}{3!} \int^x_a f'''(t)d((t - x)^3)\\
& \overset{I.B.P}{=} & \displaystyle f(a) + f'(a)(x - a) + \frac{1}{2}f''(a)(x - a)^2 + \frac{1}{3!} f'''(a)(x - a)^3 \Big|^{t = x}_{t = a} - \int^x_a (t - x)^3 f^{(4)}(t)dt\\
& = & \displaystyle f(a) + f'(a)(x -a) + \frac{1}{2} f''(a)(x - a)^2 + \frac{1}{3!}f'''(a)(x -a)^3 - \frac{1}{3!} \int^x_a (t - x)^3 f^{(4)}(t)dt\\
& \vdots &\\
f(x) & = & \displaystyle f(a) + f'(a)(x - a) + \frac{1}{2!} f''(a)(x - a)^2 + \cdots + \frac{1}{n!} f^{(n)}(a) (x - a)^n\\
&& \displaystyle + \underline{(-1)^n \frac{1}{n!} \int^x_a (t - x)^n f^{(t + 1)}(t)dt} \quad \tsorange{Rn(x)}\\
& = & \displaystyle \tssteelblue{\sum^{\infty}_{k = 0} \frac{f^{(n)}(a)}{k!} (x - a)^k} + \tsorange{\frac{(-1)^n}{n!} \int^x_a (t - x)^n f^{(n + 1)}(t)dt}\\
& = & \tssteelblue{Tn(x)} + \tsorange{Rn(x)}
\end{array}\]
\begin{proof} Taylor Inequality
\[\begin{array}{rcl}
Rn(x) & = & \displaystyle \frac{1}{n!} \int^x_a (x - t)^n f^{(n + 1)}(t)dt\\
|Rn(x)| & \leq & \displaystyle \frac{1}{n!} \int^x_a |x - t||f^{(n + 1)}(t)|dt\\
& \leq & \displaystyle \frac{1}{n!} \int^x_a |x - t|^n Mdt\\
& = & \displaystyle \frac{n}{n!} \frac{(x - a)^{n + 1}}{n + 1}\\
& = & \displaystyle \frac{M}{(n + 1)!} (x - a)^{n + 1}\\
& \leq & \displaystyle \frac{M}{(n + 1)!} |x - a|^{n + 1}
\end{array} \quad \quad \quad
\begin{array}{rcl}
\displaystyle \int^x_a |x - t|^n dt & = & \displaystyle \int^x_a (x - t)^n dt\\
& = & \displaystyle \frac{-1}{n + 1} (x - t)^{n + 1} \Big|^{t = x}_{t = a}\\
& = & \displaystyle \frac{-1}{n + 1} (0 - (x - a)^{n + 1})\\
& = & \displaystyle \frac{1}{n + 1} (x - a)^{n + 1}
\end{array}\]
\end{proof}

\begin{eg}
\begin{itemize}
\item[(a)] Approximate \(f(x) = x^{\frac{1}{3}}\) by a Taylor polynomial of degree \(2\) at \(a = 8\)

\soln
\[T_2(x) = \displaystyle \sum^2_{k = 0} \frac{f^{(k)}(2)}{k!} (x - a)^k\]
\[\begin{array}{rcl}
f(x) & = & \displaystyle x^{\frac{1}{3}}\\
f'(x) & = & \displaystyle \frac{1}{3} x^{-\frac{2}{3}}\\
f''(x) & = & \displaystyle - \frac{2}{9}x^{-\frac{5}{3}}
\end{array} \quad \quad \quad
\begin{array}{rcl}
f(8) & = & 2\\
f'(8) & = & \displaystyle \frac{1}{12}\\
f''(8) & = & \displaystyle - \frac{2}{9} \cdot \frac{1}{32} = \frac{-1}{144}
\end{array}\]
\[\displaystyle T_2 (x) = f(8) + f'(8)(x - 8) + \frac{f''(8)}{2} (x -8)^2 = 2 + \frac{1}{12} (x - 8) - \frac{1}{288}(x - 8)^2\]
\(T_2(x)\) is a quadratic (二次) fcn.
\item[(b)] How accurate is the approximation when \(7 \leq x \leq 9\)?

\soln
\[n = 2, a = 8, d = 1\]
\[\displaystyle f'''(x) = \frac{10}{27} \cdot \frac{1}{x^{\frac{8}{3}}}\]
\[\displaystyle |f'''(x)| = \Big| \frac{10}{27} \cdot \frac{1}{x^{\frac{8}{3}}} \Big| \leq \frac{10}{27} \cdot \frac{1}{7^{\frac{8}{3}}} \tssteelblue{= M} \text{ when } |x - 8| \leq 1\]
\[\displaystyle |R_2(x)| \leq \frac{M}{3!} |x - 8|^3 \leq \frac{1}{3!} \cdot \frac{10}{27} \cdot \frac{1}{7^{\frac{8}{3}}} \cdot 1^3 \Doteq 0.000344\]
\[|x - 8| \leq 1\]
\end{itemize} 
\end{eg}
\section{Review}
\begin{itemize}
\item Divergence Test\\
\(\displaystyle \sum^{\infty}_{n = 0} a_n\) converges \(\implies \displaystyle \lim_{n \to \infty} a_n = 0\)\\
(\(\displaystyle \lim_{n \to \infty} a_n \neq 0\) or doesn't exist \(\implies \displaystyle \sum^{\infty}_{n = 0}\) diverges)
\item Alternating Series Test
\[\displaystyle \sum^{\infty}_{n = 0} (-1)^n a_n \quad a_n > 0\]
\begin{itemize}
\item[(1)] \(\displaystyle \lim_{n \to \infty} a_n = 0\)
\item[(2)] \(a_n \searrow\) in \(n\)
\end{itemize}
\item Integral test
\begin{itemize}
\item[(1)] \(a_n > 0 \quad \forall n = 1, 2, \cdots\)
\item[(2)] \(a_n \searrow \) in \(a\)
\end{itemize}
\(\displaystyle \sum^{\infty}_{n = 1} a_n \neq \int^{\infty}_1 a_n dn\)\\
Both converge or diverge
\item Comparison Test
\[\displaystyle \sum a_n \quad \sum b_n \quad \quad a_n, b_n > 0\]
\begin{itemize}
\item 減法
\begin{itemize}
\item[(1)] \(\displaystyle a_n \geq b_n \quad \sum a_n \text{ converges } \implies \sum b_n\) converges
\item[(2)] \(\displaystyle a_n \geq b_n \quad \sum b_n \text{ diverges }\)\\
\(\implies \sum a_n\) diverges
\end{itemize}
\item 除法
\(\displaystyle \lim_{n \to \infty} \frac{a_n}{b_n} = L\)
\begin{itemize}
\item[(1)] \( L = 0 \quad \displaystyle \sum b_n \text{ converges } \implies \sum a_n\) converges
\item[(2)] \( L = \infty\) \quad \(\displaystyle \lim_{n \to \infty} \frac{b_n}{a_n} = 0\)
\item[(3)] \(L \neq 0\) and \(L \neq \infty \quad (0 < L < \infty) \implies \displaystyle \sum a_n \text{ and } \sum b_n\) both converge or diverge
\end{itemize}
\end{itemize}
\item Ratio and Root Test
\begin{itemize}
\item Ratio Test\\
\[\displaystyle \lim_{n \to \infty} \Big| \frac{a_{n+1}}{a_n}\Big| = L\]
\begin{itemize}
\item[(1)] \(L < 1 \quad \displaystyle \sum a_n \text{ is A.C } \implies \sum a_n\) converges \quad (i.e \(\displaystyle \sum |a_n|\) converges)
\item[(2)] \(L > 1 \quad \displaystyle \sum a_n\) diverges
\item[(3)] \(L = 1 \quad\) Inconclusive\\
\(\displaystyle \sum^{\infty}_{n = 1} \frac{1}{n} \text{ diverges } \implies L = 1\\
\displaystyle \sum^{\infty}_{n = 1} \frac{1}{n^2} \text{ converges } \implies L = 1\)
\end{itemize}
\item Root test
\[\displaystyle \lim_{n \to \infty} |a_n|^{\frac{1}{n}} = L\]
\begin{itemize}
\item[(1)] \(L < 1 \quad \displaystyle \sum a_n \text{ is A.C } \implies \sum a_n\) converges \quad (i.e \(\displaystyle \sum |a_n|\) converges)
\item[(2)] \(L > 1 \quad \displaystyle \sum a_n\) diverges
\item[(3)] \(L = 1 \quad\) Inconclusive
\end{itemize}
\end{itemize}
\item Power Series: Representation\\
\(\displaystyle \sum^{\infty}_{n = 0} c_n (x - a)^n\) depends on \(x\)\\
I.O.C is \(|x -a| \leq R\)\\
R.O.C is \(R\)\\
no negative power 
\begin{itemize}
\item[(1)] \(\displaystyle \frac{1}{1 - x} = 1 + x + x^2 + \cdots = \sum^{\infty}_{n = 0} x^n \quad |x| < 1\)
\item[(2)] \(\displaystyle \int f'(x)dx = f(x) + C\\
\int \sum (c_n (x - a)^n) dx\\
\sum (\int c_n(x - a)^n dx)\)
\end{itemize}
\item Taylor Series of \(f\) at \(a\)
\(\displaystyle \sum^{\infty}_{n = 0} \frac{f^{(n)}(a)}{n!}\)\\
when \(a = 0 \implies\) Maclaurin series 
\begin{itemize}
\item[(1)] \(\displaystyle e^x = \sum^{\infty}_{n = 0} \frac{x^n}{n!} \quad x \in R\)
\item[(2)] \(\displaystyle \sin x  = \sum^{\infty}_{n = 0} \frac{(-1)^n}{(2n + 1)!}x^{2n + 1} \quad x \in R\)
\item[(3)] \(\displaystyle \cos x = \sum^{\infty}_{n = 0} \frac{(-1)^n}{(2n)!} x^{2n} \quad x \in R\)
\end{itemize}
\end{itemize}